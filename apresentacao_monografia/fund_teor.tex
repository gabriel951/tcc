\section{Fundamentação Teórica}

\subsection{Especificidades da UnB}
\begin{frame}{IRA}
    Fórmula para cálculo do IRA, segundo \cite{manual_calouro}, é mostrada abaixo: 
    \begin{equation}
        IRA = (1 - \frac{0,6 * DT_b + 0,4 * DT_p}{DC}) * 
        (\frac{\sum_{i}P_i * CR_i * Pe_i}{\sum_{i} CR_i * Pe_i})
    \end{equation}
\end{frame}

\subsection{Modelo de Referência CRISP-DM}
\begin{frame}{Modelo de Referência CRISP-DM}
    Na pesquisa foi usado o modelo de referência CRISP-DM \cite{crispdm}. 
    
    \vspace{0.5cm}

    CRISP-DM é um modelo de referência para mineração de dados que se divide em seis
    fases: 
    \begin{itemize}
        \item Entendimento do Negócio
        \item Entendimento dos Dados
        \item Preparação dos Dados
        \item Modelagem
        \item Avaliação 
        \item Implantação
    \end{itemize}
\end{frame}

\subsection{Algoritmos de Aprendizagem de Máquina}
\begin{frame}{Algoritmos de Aprendizagem de Máquina}
    Os seguintes algoritmos de aprendizagem de máquina (veja \cite{ml_book} ou
    \cite{ml_second_book}) foram utilizados: 
    \begin{itemize}
        \item ANN
        \item \textit{Naive Bayes}
        \item \textit{Random Forest}
        \item Regressor Linear
        \item SVR
    \end{itemize}
\end{frame}

\subsection{Métricas para Avaliação dos Modelos}
\begin{frame}{Métricas para Avaliação dos Modelos}
    Métricas comuns para o problema de classificação supervisionado incluem:
    precisão, \textit{recall} e \textit{F-measure}.

    \begin{equation}
        \text{Precisão} = \frac{TP}{TP + FP} 
    \end{equation}

    \vspace{0.2cm}

    \begin{equation}
        \text{Recall} = \frac{TP}{TP + FN}
    \end{equation}

    \vspace{0.2cm}
\end{frame}

\begin{frame}{F-measure}
    \textit{F-measure} é definida como sendo a média harmônica ponderada entre
    precisão e \textit{recall.} 

    \vspace{0.2cm}

    \begin{equation}
        \text{F-measure} = (1 + \beta^2 ) * \frac{\text{precisão} * \text{recall}}
                                             {\beta^2 * \text{precisão} + \text{recall}}
    \end{equation}
\end{frame}

\begin{frame}{F-measure}
    Quando a ponderação não privilegia precisão ou \textit{recall}: 
    \begin{equation}
        \text{F-measure} = 2 * \frac{\text{Precisão} * \text{recall}}
                                    {\text{Precisão} + \text{recall}}
    \end{equation}
\end{frame}

