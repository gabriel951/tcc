\section{Fundamentação Teórica}
% todo: if i want put section with unb peculiarities
\subsection{Modelo de Referência CRISP-DM}
\begin{frame}{Modelo de Referência CRISP-DM}
    Na pesquisa foi usado o modelo de referência CRISP-DM. 
    
    \vspace{0.5cm}

    CRISP-DM é um modelo de referência para mineração de dados que se divide em seis
    fases: 
    \begin{itemize}
        \item Entendimento do Negócio
        \item Entendimento dos Dados
        \item Preparação dos Dados
        \item Modelagem
        \item Avaliação 
        \item Implantação
    \end{itemize}
\end{frame}

\subsection{Algoritmos de Aprendizagem de Máquina}
\begin{frame}{Algoritmos de Aprendizagem de Máquina}
    Os seguintes algoritmos de aprendizagem de máquina foram utilizados: 
    \begin{itemize}
        \item ANN
        \item Naive Bayes
        \item Random Forest
        \item Regressor Linear
        \item SVR
    \end{itemize}
\end{frame}

\subsection{Métricas para Avaliação dos Modelos}
\begin{frame}{Métricas para Avaliação dos Modelos}
    Métricas comuns para o problema de classificação supervisionado incluem:
    precisão, recall e F-measure. 


    \begin{equation}
        \text{Precisão} = \frac{TP}{TP + FP} 
    \end{equation}

    \vspace{0.2cm}

    \begin{equation}
        \text{Recall} = \frac{TP}{TP + FN}
    \end{equation}

    \vspace{0.2cm}

    F-measure é definida como sendo a média harmônica entre precisão e recall. 
\end{frame}

