\begin{frame}{Banca do TCC}
    Membros da Banca: 
    \begin{itemize}
        \item Profª. Drª. Letícia Lopes Leite
        \item Profª. Drª. Maria de Fátima Ramos Brandão
        \item Prof. Dr. Thiago de Paulo Faleiros
    \end{itemize}
    \vspace{0.5cm}
    Obrigado por virem!
\end{frame}

% section - Introdução
\section{Introdução}
\subsection{Definição do Problema}
\begin{frame}{Problemas Causados pela Evasão}
    Evasão em universidades brasileiras traz desperdícios acadêmicos, sociais e
    econômicos.  

    \vspace{0.5cm}

    A Universidade de Brasília teve prejuízo estimado em 95,6 milhões em 2014
    \cite{correio}.
\end{frame}

\begin{frame}{Abordagem da UnB e Problemas}
    A Unb adota a seguinte abordagem: 
    \begin{itemize}
        \item Separar alunos em condição dos demais.
        \item Ter alunos em condição supervisionados por orientador. 
    \end{itemize}

    \vspace{0.5cm}

    Problemas da abordagem da UnB: 
    \begin{itemize}
        \item Alunos (amostra diversificada) separados em apenas dois grupos.
        \item UnB age apenas quando aluno já está em condição.
    \end{itemize}
\end{frame}

\subsection{Proposta de Solução}
\begin{frame}{Proposta de Solução}
    Utilizar dados descaracterizados para criação de um sistema previsor capaz de 
    identificar alunos em risco de serem desligados. 
\end{frame}

\begin{frame}{Entendendo a Saída}
    Sistema previsor fornece uma tripla $(v_1, v_2, v_3)$ de valores entre 0 e 1 que
    somam 1. 

    \vspace{0.5cm}

    $v_1, v_2, v_3$ indicam respectivamente a chance do aluno se graduar, ser
    desligado ou migrar de curso. 
\end{frame}

\begin{frame}{Vantagens}
    Sistema permitiria à UnB agir com antecedência e flexibilidade: 
    \begin{itemize}
        \item Ações podem ser tomadas antes de um aluno entrar em condição. 
        \item Ações podem ser tomadas de acordo com o risco apresentado por um aluno.
    \end{itemize}
\end{frame}

\subsection{Objetivos}
\begin{frame}{Objetivos}
    São objetivos da pesquisa: 
    \begin{itemize}
        \item Induzir modelos de aprendizagem de máquina que indiquem a possibilidade
            do aluno concluir, evadir ou migrar de curso. 
        \item Avaliar a viabilidade da abordagem para a área de computação. Em uma
            segunda fase, aplicar a metodologia para os demais cursos. 
    \end{itemize}
\end{frame}

\begin{frame}{Cursos Estudados}
    Cursos considerados: 
    \begin{itemize}
        \item Ciência da Computação
        \item Engenharia de Computação
        \item Engenharia de Controle e Automação (Engenharia Mecatrônica)
        \item Engenharia de Redes
        \item Engenharia de Software
        \item Licenciatura em Computação
    \end{itemize}
\end{frame}
