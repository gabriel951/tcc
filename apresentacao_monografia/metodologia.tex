\section{Metodologia} 
\subsection{Levantamento do Estado da Arte}
\begin{frame}{Levantamento do Estado da Arte}
    Levantamento do estado da arte, de modo a compreender: 
    \begin{itemize}
        \item Quais fatores considerar \cite{dropout_finland}.
        \item Como aplicar técnicas de aprendizagem de máquina \cite{adeodato}.
        \item Peculiaridades da UnB \cite{hoed_fatores}.
    \end{itemize}
\end{frame}

\begin{frame}{Artigo de Adeodato e Barros \cite{adeodato}}
    Mineração de dados para avaliar risco do aluno ficar mais tempo que o previsto na
    universidade.

    \vspace{0.5cm}

    Viabilidade econômica da implementação de um processo de aconselhamento. 
\end{frame}

\subsection{Obtenção e Utilização dos Dados}
\begin{frame}{Obtenção e Utilização dos Dados}
    Informações descaracterizadas de alunos de graduação da UnB. Dados pessoais e de
    desempenho acadêmico. 

    \vspace{0.5cm}

    Considerou-se apenas alunos que ingressaram a partir de 2000 e saíram até 2016.
\end{frame}

\subsection{Seleção Preliminar de Atributos}
\begin{frame}{Seleção Preliminar de Atributos}
    Atributos pessoais considerados em uma análise inicial: 
    \begin{itemize}
        \item Cotista (ou não)
        \item Curso 
        \item Forma de Ingresso
        \item Idade
        \item Raça 
        \item Sexo 
        \item Tipo da Escola
    \end{itemize}
\end{frame}

\begin{frame}{Seleção Preliminar de Atributos}
    Atributos relativos ao desempenho acadêmico: 
    \begin{itemize}
        \item Coeficiente de Melhora Acadêmica
        \item Indicador de Aluno em Condição
        \item Média do Período
        \item Posição em relação ao semestre que ingressou
        \item Quantidade de créditos já integralizados
        \item Taxa de Aprovação, Reprovação e Trancamento
        \item Taxa de aprovação na disciplina mais difícil do semestre
    \end{itemize}
\end{frame}

\begin{frame}{Eliminação de Atributos Devido a Missing Values}
    Optou-se por eliminar atributos com percentagem de \textit{missing values} maior que 40\%.
        
    \vspace{0.5cm}
    
    Assim, atributos raça e tipo da escola foram eliminados. 
\end{frame}

\subsection{Estatística Descritiva}
\begin{frame}{Eliminação de Outliers}
    Eliminaram-se \textit{outliers}. 233 estudantes foram eliminados do espaço amostral,
    que ficou com 4536 estudantes.  
\end{frame}

\begin{frame}{Mudança na Base de Dados}
    Atributos variavam significativamente de acordo com: 
    \begin{itemize}
        \item Curso do Aluno
        \item Idade do Aluno
    \end{itemize}
\end{frame}

\begin{frame}{Mudança na Base de Dados}
    Decisão de particionar a base de dados original em quatro bases de dados:
    \begin{itemize}
        \item Alunos Jovens da FT 
            \begin{itemize}
                \item Engenharia de Redes
                \item Engenharia Mecatrônica
            \end{itemize}

        \item Alunos Jovens de Licenciatura
            \begin{itemize}
                \item Licenciatura em Computação
            \end{itemize}

        \item Alunos Jovens de Computação
            \begin{itemize}
                \item Ciência da Computação
                \item Engenharia de Software
                \item Engenharia de Computação
            \end{itemize}
        \item Alunos Seniores
            \begin{itemize}
                \item Todos os cursos
            \end{itemize}
    \end{itemize}
\end{frame}

\begin{frame}{Alunos - Forma de Saída}
    \begin{figure}[!ht]
        \centering
        \includegraphics[width = 7.5cm]{way_out_all.png}
    \end{figure}
\end{frame}

\begin{frame}{Alunos Jovens da FT - Forma de Saída}
    \begin{figure}[!ht]
        \centering
        \includegraphics[width = 7.5cm]{way_out_yng_ti.png}
    \end{figure}
\end{frame}

\begin{frame}{Alunos Jovens da Licenciatura - Forma de Saída}
    \begin{figure}[!ht]
        \centering
        \includegraphics[width = 7.5cm]{way_out_yng_lic.png}
    \end{figure}
\end{frame}

\begin{frame}{Alunos Jovens da Computação - Forma de Saída}
    \begin{figure}[!ht]
        \centering
        \includegraphics[width = 7.5cm]{way_out_yng_comp.png}
    \end{figure}
\end{frame}

\begin{frame}{Alunos Seniores - Forma de Saída}
    \begin{figure}[!ht]
        \centering
        \includegraphics[width = 7.5cm]{way_out_old.png}
    \end{figure}
\end{frame}

\subsection{Redução de Atributos}
\begin{frame}{Eliminação de Atributos Relacionados}
    Utilizou-se o teste de Kendall para eliminar atributos que apresentassem mais de
    80\% de correlação. 

    \vspace{0.5cm}

    Taxa de aprovação e taxa de reprovação estavam fortemente relacionados, de modo
    que eliminou-se o atributo taxa de reprovação. 
\end{frame}

\begin{frame}{Eliminação de Atributos Irrelevantes}
    Eliminação de atributos irrelevantes usando árvores de decisão: 
    \begin{itemize}
        \item Para alunos jovens da licenciatura, atributo curso não era relevante. 
        \item Para alunos seniores, os atributos curso, cota e taxa de trancamento
            não foram considerados relevantes. 
        \item Para outras bases de dados, nenhum atributo foi classificado como
            irrelevante. 
    \end{itemize}
\end{frame}

\subsection{Aprendizagem de Máquina e Ajuste de Parâmetros}
\begin{frame}{Ajuste de Parâmetros}
    Fez-se ajuste de parâmetros para:
    \begin{itemize}
        \item ANN
        \item \textit{Naive Bayes}
        \item SVR
    \end{itemize}

    \vspace{0.5cm}

    Para os demais algoritmos, utilizou-se o \textit{default} da biblioteca
    \texttt{scikit-learn}.
\end{frame}

\subsection{Divisão em Treino e Teste}
\begin{frame}{Divisão em Treino e Teste}
    Dados de Treino: Alunos que ingressaram de 2000 até 2009.

    \vspace{0.5cm}

    Dados de Teste: Alunos que ingressaram de 2010 até 2016.
\end{frame}

\subsection{Divisão em Semestres}
\begin{frame}{Necessidade da Divisão em Semestres}
    Para o problema de negócio considerado, sistema previsor deve ser capaz de
    calcular o risco de alunos evadirem tanto para alunos no início do curso quanto
    para estudantes mais adiantados. 

    \vspace{0.5cm}

    Alguns atributos dos alunos, como a taxa de aprovação, mudam a cada semestre.
\end{frame}

\begin{frame}{Funcionamento da Divisão em Semestres}
    Modelos são induzidos separadamente para cada
    semestre: 
    \begin{enumerate}
        \item Inicialmente, modelos induzidos com dados do 1º semestre dos
             aluno do conjunto de treino e avaliados com dados do 1º semestre do
             conjunto de teste. 
         \item Repete-se o procedimento para os dados do 2º semestre dos alunos e
             assim por diante.
    \end{enumerate}
\end{frame}

\subsection{Avaliação de Desempenho}
\begin{frame}{Processo de Avaliação de Desempenho}
    Processo: 
    \begin{enumerate}
        \item Cada modelo induzido gera, para cada aluno em cada semestre ativo, uma
            tripla que indica a possibilidade do aluno concluir, evadir ou migrar. 
        \item Maior valor da tripla é usado como sendo a previsão do modelo. 
        \item Compara-se a previsão com o que realmente aconteceu com o aluno, de
            modo a verificar se o modelo acertou ou errou. 
    \end{enumerate}
\end{frame}

\begin{frame}{Diagrama para Avaliação do Desempenho}
    \begin{figure}[!ht]
        \centering
        \includegraphics[width = 12cm]{images/fmeasure_new_diagram.png}
    \end{figure}
\end{frame}

\begin{frame}{Avaliação do Desempenho}
    Repete-se o processo descrito anteriormente para cada um dos semestres estudados.
    Assim, um determinado modelo tem, para uma base de dados, vários valores de
    \textit{F-measure} calculados, um para cada semestre. 
\end{frame}

\begin{frame}{Avaliação de Desempenho}
    Para sumarizar o desempenho do algoritmo, calculou-se a média das
    \textit{F-measures} de cada semestre: 
    \begin{equation}
        \text{F-measure média} = \frac{\sum\limits_{1 \le i \le n} \text{Fmeasure}_i}
                                    {n}
    \end{equation}
\end{frame}

\begin{frame}{Avaliação do Desempenho}
    Modelos tem desempenho comparado entre si e com o ZeroR. 

    \vspace{0.5cm}

    ZeroR é um classificador simples que sempre prevê a classe majoritária. 
\end{frame}

