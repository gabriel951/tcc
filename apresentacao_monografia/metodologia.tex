\section{Metodologia} 
\subsection{Obtenção e Utilização dos Dados}
\begin{frame}{Obtenção e Utilização dos Dados}
    Informações descaracterizadas de alunos de graduação da UnB. Dados sociais e de
    desempenho acadêmico. 

    \vspace{0.5cm}

    Considerou-se apenas alunos que ingressaram a partir de 2000 e saíram até 2016.
\end{frame}

\subsection{Seleção de Atributos}

\begin{frame}{Seleção Preliminar de Atributos}
    Atributos sociais considerados em uma análise inicial: 
    \begin{itemize}
        \item Cotista (ou não)
        \item Curso 
        \item Forma de Ingresso
        \item Idade
        \item Raça 
        \item Sexo 
        \item Tipo da Escola
    \end{itemize}
\end{frame}

\begin{frame}{Seleção Preliminar de Atributos}
    Atributos relativos ao desempenho acadêmico: 
    \begin{itemize}
        \item Coeficiente de Melhora Acadêmica
        \item Indicador de Aluno em Condição
        \item Média do Período
        \item Posição em relação ao semestre que ingressou
        \item Quantidade de créditos já integralizados
        \item Taxa de Aprovação, Reprovação e Trancamento
        \item Taxa de aprovação na disciplina mais difícil do semestre
    \end{itemize}
\end{frame}

\begin{frame}{Eliminação de Atributos Devido a Missing Values}
    Optou-se por eliminar atributos com percentagem de missing values maior que 40\%.
        
    \vspace{0.5cm}
    
    Assim, atributos raça e tipo da escola foram eliminados. 
\end{frame}


\subsection{Estatística Descritiva}
\begin{frame}{Eliminação de Outliers}
    Eliminaram-se outliers. 233 estudantes foram eliminados do espaço amostral,
    ficando com 4536 estudantes.  
\end{frame}
