\section{Resultados}
\subsection{Resultados dos Ajuste de Parâmetros}
\begin{frame}{Resultados do Ajuste de Parâmetros para ANN}
    Função de ativação: ``relu''

    \vspace{0.5cm}

    Melhores configurações para ANN de acordo com a base de dados: 
    \begin{table}
    \begin{center}
        \begin{tabular}[c]{| c | c | c |}
        \hline
        \textbf{Base de Dados} & \textbf{Neurônios} & \textbf{Aprendizagem} \\
        \hline
        Jovens - FT & 100 & 0.001 \\
        \hline
        Jovens - Licenciatura & 36 & 1.0 \\
        \hline
        Jovens - Computação & 36 & 0.001 \\
        \hline
        Seniores & 24  & 0.7 \\
        \hline
    \end{tabular}
    \end{center}
    \end{table}
\end{frame}

\begin{frame}{Resultados do Ajuste de Parâmetros para SVR}
    Melhores configurações para SVR de acordo com a base de dados: 
    \begin{table}
    \begin{center}
    \begin{tabular}[c]{| c | c | c |}
        \hline
        \textbf{Base de Dados} & \textbf{Kernel} & \textbf{Penalização} \\
        \hline
        Jovens - FT & linear & 1.0 \\
        \hline
        Jovens - Licenciatura & linear & 1.0 \\
        \hline
        Jovens - Computação & rbf & 1.0 \\
        \hline
        Seniores & linear & 1.0 \\
        \hline
    \end{tabular}
    \end{center}
    \end{table}
\end{frame}

\begin{frame}{Resultados do Ajuste de Parâmetros para Naive Bayes}
    Melhores configurações para Naive Bayes de acordo com a base de dados: 
    \begin{table}
    \begin{center}
    \begin{tabular}[c]{| c | c | c |}
        \hline
        \textbf{Base de Dados} & \textbf{Distribuição dos Atributos} \\
        \hline
        Jovens - FT & Gaussiana \\
        \hline
        Jovens - Licenciatura & Bernoulli \\
        \hline
        Jovens - Computação & Multinomial \\
        \hline
        Seniores & Gaussiana \\
        \hline
    \end{tabular}
    \end{center}
    \label{conf_nb}
    \end{table}
\end{frame}

\subsection{F-measures Obtidos}
%\begin{frame}{Resultado - Alunos Jovens da FT}
%    \textit{F-measure} média para alunos jovens da FT: 
%    \begin{table}
%    \begin{center}
%    \begin{tabular}[c]{| c | c |}
%        \hline
%        \textbf{Algoritmo} & \textbf{F-measure} \\
%        \hline
%        ANN              & 0.76 \\
%        \hline
%        Naive Bayes      & 0.56 \\
%        \hline
%        Random Forest    & 0.73 \\
%        \hline
%        Regressor Linear & 0.80 \\
%        \hline
%        SVR              & 0.76 \\
%        \hline
%        ZeroR            & 0.64 \\
%        \hline
%    \end{tabular}
%    \end{center}
%    \end{table}
%\end{frame}
%
%\begin{frame}{Resultado - Alunos Jovens da Licenciatura}
%    \textit{F-measure} média para alunos jovens da licenciatura: 
%    \begin{table}
%    \begin{center}
%    \begin{tabular}[c]{| c | c |}
%        \hline
%        \textbf{Algoritmo} & \textbf{F-measure} \\
%        \hline
%        ANN              & 0.85 \\
%        \hline
%        Naive Bayes      & 0.76 \\
%        \hline
%        Random Forest    & 0.85 \\
%        \hline
%        Regressor Linear & 0.86 \\
%        \hline
%        SVR              & 0.82 \\
%        \hline
%        ZeroR            & 0.70 \\
%        \hline
%    \end{tabular}
%    \end{center}
%    \end{table}
%\end{frame}
%
%\begin{frame}{Resultado - Alunos Jovens da Computação}
%    \textit{F-measure} média para alunos jovens da computação: 
%    \begin{table}
%    \begin{center}
%    \begin{tabular}[c]{| c | c |}
%        \hline
%        \textbf{Algoritmo} & \textbf{F-measure} \\
%        \hline
%        ANN              & 0.74 \\
%        \hline
%        Naive Bayes      & 0.65 \\
%        \hline
%        Random Forest    & 0.76 \\
%        \hline
%        Regressor Linear & 0.77 \\
%        \hline
%        SVR              & 0.70 \\
%        \hline
%        ZeroR            & 0.60 \\
%        \hline
%    \end{tabular}
%    \end{center}
%    \end{table}
%\end{frame}
%
%\begin{frame}{Resultado - Alunos Seniores}
%    \textit{F-measure} média para alunos seniores: 
%    \begin{table}
%    \begin{center}
%    \begin{tabular}[c]{| c | c |}
%        \hline
%        \textbf{Algoritmo} & \textbf{F-measure} \\
%        \hline
%        ANN              & 0.62 \\
%        \hline
%        Naive Bayes      & 0.28 \\
%        \hline
%        Random Forest    & 0.70 \\
%        \hline
%        Regressor Linear & 0.75 \\
%        \hline
%        SVR              & 0.79 \\
%        \hline
%        ZeroR            & 0.61 \\
%        \hline
%    \end{tabular}
%    \end{center}
%    \end{table}
%\end{frame}

\begin{frame}{Resultados - Síntese}
    \textit{F-measure} dos modelos de acordo com a base de dados: 
    \begin{table}
    \begin{center}
    \begin{tabular}[c]{| c | c | c | c | c |}
        \hline
        \textbf{Algoritmo} & \textbf{Sen} & \textbf{J - FT} & 
        \textbf{J - Lic} & \textbf{J - Comp} \\
        \hline
        ANN              & 0.62 & 0.76 & 0.85 & 0.74 \\
        \hline
        Naive Bayes      & 0.28 & 0.56 & 0.76 & 0.65 \\
        \hline
        Random Forest    & 0.70 & 0.73 & 0.85 & 0.76 \\
        \hline
        Regressor Linear & 0.75 & \textbf{0.80} & \textbf{0.86} & \textbf{0.77} \\
        \hline
        SVR              & \textbf{0.79} & 0.76 & 0.82 & 0.70 \\
        \hline
        ZeroR            & 0.61 & 0.64 & 0.70 & 0.60 \\
        \hline
    \end{tabular}
    \end{center}
    \end{table}
\end{frame}

\subsection{Discussão dos Resultados}
\begin{frame}{Mau Desempenho do Naive Bayes}
    Mau desempenho do Naive Bayes justifica-se pelos atributos passados
    não poderem ser considerados condicionalmente independentes. 
\end{frame}

\begin{frame}{Bom Desempenho dos Modelos, Especialmente o Regressor Linear}
    De forma geral, algoritmos conseguiram ter melhor resultado que o ZeroR.  

    \vspace{0.5cm}

    Bom desempenho obtido pelo regressor linear: \textit{F-measure} em torno de 0.795. 
\end{frame}
