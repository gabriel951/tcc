\documentclass[12pt]{article}

\usepackage{sbc-template}

%black magic to break urls
\usepackage{graphicx,url}
\makeatletter
\g@addto@macro{\UrlBreaks}{\UrlOrds}
\makeatother

%\usepackage[brazil]{babel}   
\usepackage[latin1]{inputenc}  
     
\sloppy

\title{A Machine Learning Predictive System to Identify Students in Risk of Dropping
Out of College}

\author{Gabriel F. Silva\inst{1}, Marcelo Ladeira \inst{1}}

\address{Departamento de Ci�ncia da Computa��o -- Universidade de Bras�lia 
\email{mladeira@unb.br, gabrieltiburciouaiso@gmail.com}
}

\begin{document} 

\maketitle

\begin{abstract}
% problem definition and current approach
The University of Bras�lia (UnB) suffers from a big problem of student drop out, which has
academics, economics and social consequences. UnB's approach to the problem consist
of separating it's students in two groups: those that are in risk of dropping out
and those that aren't and counsel the students in the former group.  % my proposal
This paper describes the development of a predictive system capable of indicating the
risk of a student dropping out. This way, UnB could act before it's too late and
also act according to the risk presented by a student. 

% methodology - todo: put that students are from computation and are undergrad
For the development of the predictive system, data of students that entered and left
UnB from 2000 to 2016 was used. The data does not contain student identification.
Machine learning (ML) models were trained and their performance was analysed on test
data. ML models studied were Naive Bayes, ANN, SVR, Linear Regressor and Random
Forests).

% results/conclusion
Machine Learning models got, in general, good performance. The best performance came
from the linear regressor. Results obtained indicate potential in using machine
learning to predict the risk of students dropping out of university. 
\end{abstract}

\section{Introduction}
Student dropping out of brazilian universities are a significant problem, with
academic, social and economic consequences. University of Brasilia (UnB) is no
exception, being significantly affected by the problem \footnote{UnB had a money lost
estimated in 96.5 millions, according to the brazilian press Correio Brasiliense
\cite{correio}}.  
\par This paper describes a possible approach for the problem: the development of a
predictive analysis system that estimates the risk of a given student dropping
out of college. In case of success, the system would allow the university to take
measures with precedence (before it's too late) and with flexibility (according to
the risk presented by a student). 
\par The system outputs a triple $(v_1, v_2, v_3)$, with all values 
between 0 and 1. The values indicate, in this order, the chance of the given student
graduate, evade or migrate from the course he's in. 
\par The rest of the article is organized in the following structure: in the next
section, the methods used are thoroughly explained; after that, the good results
obtained are presented and discussed; finally, conclusion is made and ideas for
future work are exposed.  

\section{Methods}
In this section, the methodology used is discussed. First, the data obtained is
described, along with the train and test division. After that, feature selection is
explained. Next, the division by semester is motivated. Afterwards, the machine
learning models used are listed, with their parameters configuration.  Finally, the
performance measure used is described.

\subsection{Data, Data Division and Train and Test Division}
The data used is from undergraduate students that entered and left the university
from 2000 to 2016. To simplify the analysis, only courses in areas related to
computer science were considered. Accordingly, the courses considered were: computer
science (bachelor degree), computing (teaching degree), computer engineering,
mechatronics engineering, network engineering and software engineering. 
\par In exploratory data analysis, it was possible to observe that the features
varied significantly with the course of the student being considered. Another useful
information obtained was that the proportion of students capable of graduating
depends on the age the student enters university: students that enter older in
university are less likely to graduate. These observations led to the decision of
partitioning the data in four distinct databases:  

\begin{itemize}
    \item Seniors Students: All students with more than 30 years. 
    \item Young Students from FT: Contain all students that entered in UnB with 30
        years or less and course mechatronics engineering or network engineering.
        This courses have the distinction of being associated with FT (Faculty of
        Technology). 
    \item Young Students from computing: Contain all students that entered in UnB
        with 30 years or less and course computing. Computing is a teaching degree in
        UnB, meaning that the students from the course are prepared to be teachers of
        computer science. Another peculiarity is the fact that it's the only night
        course from the ones we are consireing. 
    \item Young Students from computer science: Contain all students that entered in
        UnB with 30 years or less and course computer science (bachelor degree),
        computer engineering or software engineering. 
\end{itemize}

\par In order to train the machine learning models, data was partitioned in the
training set and the test set. As indicated in \cite{adeodato}, a realistic division
of the data should be an ``Out of Sample'' division, in which the training set is
composed by the firsts instances and the test set from the last ones. This way, we
get a realistic scenario in which the ML models are trained to be applied in a future
time. With this in mind, the training set was composed of students that ingressed in
university from 2000 to 2009 while the test set was composed of students that
ingressed in university from 2010 to 2016. 

\subsection{Features and Feature Selection}
After initial research (see \cite{adeodato} or \cite{dropout_finland}) on which
features should be included for the ML models to train, the following social features were
selected initially: 

\begin{itemize}
    \item Age
    \item Course
    \item Quota 
    \item Race
    \item Sex
    \item Type of School 
    \item Way In
\end{itemize}

In addition to social features, academic features were, obviously, also considered: 
\begin{itemize}
    \item Amount of Credits Done
    \item Average Grades in a Semester
    \item Pass rate on the hardest subject of a semester
    \item Indicator of Condition: UnB classifies students as ``in condition'' or
        not ``�n condition'' and this classification (see \cite{manual_calouro}) is
        related to the risk of students dropping out. This boolean variable indicates
        if the student is ``in condition'' or not. 
    \item Pass Rate, Fail Rate and Drop Rate: indicate, respectively, the percentage
        of the subjects a given student pass, fails and drop. 
    \item Position in relation to fellow students: for a given student $S$ indicates,
        from all students of the same year, semester and course of $S$, how many have
        higher grades than $S$.
    \item Rate of Academic Improvement: reason between grades of a student in current
        semester by the grades of a student in the previous one. 
\end{itemize}

Feature selection was done afterwards. To avoid redundancy, a Kendall test
(see \cite{kendall}) was applied to check if any two of the attributes had very strong
correlation. Because the test indicated that the fail rate and the pass rate were
significantly related, a decision was made to consider only the pass rate (this
strong correlation was verified for all 4 databases). 
\par Moreover, to check if all the features were really necessary, decision trees
were employed: features that didn't appear as part of a decision tree were not
considered in further analysis. The results indicated that, for young students of
computing the feature course was irrelevant (which makes sense, since all students
from this group have the same courses). For seniors students, the features course,
quota and drop rate were considered irrelevant. For the other 2 databases, no feature
was considered irrelevant.

\subsection{The Necessary Semester Division}
The predictive system, for the business problem here considered, must be capable of
calculating the risk of students evading for students in the beginning of the course
and for students in the end of the course. In relation to that, students academic
features change from semester to semester. 
\par This points that a necessary semester division must be carried out. What happens
is that the machine learning models are trained separately semester by semester.
Initially, the ML models train on the train dataset containing features from the
1� semester of the students and are tested on the test dataset containing features
from the 1� semester of the students. Next, this procedure is repeated for the 2�
semester and so it goes. 

\subsection{Machine Learning Models}
The machine learning models used in this research were: ANN, linear regressor, Naive
Bayes, random forests, SVR. For more information on this models, we recommend the
excellent books \cite{ml_book} and \cite{ml_second_book}. To establish a baseline,
the ZeroR model was considered. The Python programming language was used, combined
with the scikit-learn library \cite{sklearn}. 
\par Preliminary studies were made to determine the best configurations for the ANN,
the Naive Bayes and the SVR parameters. For the other machine learning models, the
default configurations were used. The results varied according to the
databases, and are shown on Tables \ref{ann_conf}, \ref{svr_conf}, \ref{nb_conf}. 

\begin{table}
\begin{center}
\begin{tabular}[c]{| c | c | c |}
    \hline
    \textbf{Database} & \textbf{Hidden Layer Size} & \textbf{Learning Rate} \\
    \hline
    Seniors Students & 24  & 0.7 \\
    \hline
    Young Students from FT & 100 & 0.001 \\
    \hline
    Young Students from Computing & 36 & 1.0 \\
    \hline
    Young Students from Computer Science & 36 & 0.001 \\
    \hline
\end{tabular}
\end{center}
\caption{ANN's configuration, according to the database}
\label{ann_conf}
\end{table}

\begin{table}
\begin{center}
\begin{tabular}[c]{| c | c | c |}
    \hline
    \textbf{Database} & \textbf{Kernel Type} & \textbf{Penalty Parameter} \\
    \hline
    Seniors Students & linear & 1.0 \\
    \hline
    Young Students from FT & linear & 1.0 \\
    \hline
    Young Students from Computing & linear & 1.0 \\
    \hline
    Young Students from Computer Science & rbf & 1.0 \\
    \hline
\end{tabular}
\end{center}
\caption{SVR's configuration, according to the database}
\label{svr_conf}
\end{table}

\begin{table}
\begin{center}
\begin{tabular}[c]{| c | c | c |}
    \hline
    \textbf{Database} & \textbf{Feature Distribution} \\
    \hline
    Seniors Students & Gaussian \\
    \hline
    Young Students from FT & Gaussian \\
    \hline
    Young Students from Computing & Bernoulli \\
    \hline
    Young Students from Computer Science & Multinomial \\
    \hline
\end{tabular}
\end{center}
\caption{Naive Bayes configuration, according to the database}
\label{nb_conf}
\end{table}

\subsection{Performance Measures}
As is standard practice in machine learning, the ML models trained on the train dataset and
had their perfomance measured in a test dataset, with unseen data. This process was
done for each one of the semester studied and works as explained next. 
\par Each machine learning model generates, for each student in each semester, a
triple that indicates the assessed possibility of a student graduating, evading or
migrating. The highest value of the triple is used as the model prediction. This
prediction is then compared to what really happened to the student. 
\par The metric used to evaluate the performance of the models was the F-measure. A
given machine learning model has, for a given database, values of F-measure
calculated, one for each semester. To summarize the performance of the model for a
database, the mean of the F-measures was calculated. 

\section{Results And Discussion}
The results are shown on the tables 


\begin{table}
\begin{center}
\begin{tabular}[c]{| c | c |}
    \hline
    \textbf{Modelo} & \textbf{F-measure} \\
    \hline
    ANN              & 0.76 \\
    \hline
    Naive Bayes      & 0.56 \\
    \hline
    Random Forest    & 0.73 \\
    \hline
    Regressor Linear & 0.80 \\
    \hline
    SVR              & 0.76 \\
    \hline
    ZeroR            & 0.64 \\
    \hline
\end{tabular}
\end{center}
\caption{F-measure M�dia por Modelo, para Alunos Jovens da FT}
\label{fmeasure_ft}
\end{table}

\begin{table}
\begin{center}
\begin{tabular}[c]{| c | c |}
    \hline
    \textbf{Modelo} & \textbf{F-measure} \\
    \hline
    ANN              & 0.85 \\
    \hline
    Naive Bayes      & 0.76 \\
    \hline
    Random Forest    & 0.85 \\
    \hline
    Regressor Linear & 0.86 \\
    \hline
    SVR              & 0.82 \\
    \hline
    ZeroR            & 0.70 \\
    \hline
\end{tabular}
\end{center}
\caption{F-measure M�dia por Modelo, para Alunos Jovens de Licenciatura}
\label{fmeasure_lic}
\end{table}

\begin{table}
\begin{center}
\begin{tabular}[c]{| c | c |}
    \hline
    \textbf{Modelo} & \textbf{F-measure} \\
    \hline
    ANN              & 0.74 \\
    \hline
    Naive Bayes      & 0.65 \\
    \hline
    Random Forest    & 0.76 \\
    \hline
    Regressor Linear & 0.77 \\
    \hline
    SVR              & 0.70 \\
    \hline
    ZeroR            & 0.60 \\
    \hline
\end{tabular}
\end{center}
\caption{F-measure M�dia por Modelo, para Alunos Jovens da Computa��o}
\label{fmeasure_comp}
\end{table}

\begin{table}
\begin{center}
\begin{tabular}[c]{| c | c |}
    \hline
    \textbf{Modelo} & \textbf{F-measure} \\
    \hline
    ANN              & 0.62 \\
    \hline
    Naive Bayes      & 0.28 \\
    \hline
    Random Forest    & 0.70 \\
    \hline
    Regressor Linear & 0.75 \\
    \hline
    SVR              & 0.79 \\
    \hline
    ZeroR            & 0.61 \\
    \hline
\end{tabular}
\end{center}
\caption{F-measure M�dia por Modelo, para Alunos Seniores}
\label{fmeasure_old}
\end{table}


\section{Conclusion}

\bibliographystyle{sbc}
\bibliography{article.bib}

\end{document}
