%http://aprender_unb.br/course/view.php?id=25
%http://www.cs.berkeley.edu/~jrs/speaking.html
Este capítulo oferece sugestões de como fazer a apresentação do trabalho. Uma
apresentação é necessária ao final do curso, é nela que se mostra os resultados
obtidos de forma resumida e, preferencialmente, simplificada. Embora o ``verdadeiro''
resultado seja o texto técnico, que de fato representa a contribuição científica
obtida, a apresentação serve para divulgar seus resultados e incentivar outros a
se interessarem por seu trabalho.

\section{Falando em Público}
A ideia de uma apresentação não é mostrar todos os detalhes técnicos ou tentar
impressionar o público com seu conhecimento. O objetivo é apresentar suas principais
ideias de forma intuitiva, de modo que os presentes entendam o que foi feito e
se interessem em conhecer as minúcias lendo o texto técnico.

Vale lembrar que embora você veja seu trabalho como extremamente interessante,
geralmente seu público [ainda] não acha, e provavelmente têm coisas melhores para fazer...
É preciso atrair e manter a atenção deles, bem como garantir que eles se lembrem
do que foi dito (pelo menos da ideia principal).

Algumas noções importantes:
\begin{description}
\item[Motivação:] Qual o problema e por que ele merece atenção?
\item[Ideia Principal:] Clara e explicitamente especificada.
\item[Exemplos:] A melhor forma de passar informações (ilustram motivação,
funcionamento, casos extremos, limitações, etc.).
\end{description}

\emph{Slides} são uma excelente ferramenta \textbf{de apoio} ao apresentador, mas
muitas vezes tomam vida própria e se tornam o elemento principal. É essencial,
embora um pouco difícil,  evitar a ``morte por Powerpoint''\footnote{%
\url{http://www.smallbusinesscomputing.com/biztools/article.php/684871/Death-By-Powerpoint.htm}}.

Existem muitas sugestões para fazer uma boa apresentação\footnote{\url{https://hbr.org/2014/11/how-to-give-a-stellar-presentation}},
por exemplo, imitar um bom apresentador\footnote{\url{https://www.youtube.com/watch?v=2-ntLGOyHw4}}, boas práticas na elaboração de slides\footnote{\url{https://www.youtube.com/watch?v=Iwpi1Lm6dFo}}, como organizar o conteúdo de um slide\footnote{\url{https://hbr.org/2012/10/do-your-slides-pass-the-glance-test}} (ou mesmo ``vida após a morte''\footnote{\url{https://www.youtube.com/watch?v=lpvgfmEU2Ck}}). Entretanto, as duas noções mais importantes são: você nunca se prepara demais para fazer uma apresentação, e a única regra de uma apresentação é a
de atenção\footnote{\url{http://finiteattentionspan.wordpress.com/2009/11/02/the-only-rule-about-giving-presentations-that-matters-is-the-rule-of-attention}}.

Olivia Mitchell sugere as seguintes formas de manter a atenção da platéia%
\footnote{\url{http://www.speakingaboutpresenting.com/content/7-ways-audience-attention-presentation}}:
\begin{enumerate}
\item Fale sobre algo que interesse a platéia.
\item Diga porque deveriam prestar atenção.
\item Não apresente algo muito fácil ou muito difícil.
\item ``Mudanças'' prendem a atenção.
\item Conte estórias.
\item Faça pausas.
\item Seja breve.
\end{enumerate}

Demonstrações ao vivo são impressionantes, desde que funcionem corretamente e não
evidenciem as limitações do seu trabalho. Lembre-se que eventos importantes são,
em sua maioria, regidos pela \emph{\href{http://www.humornaciencia.com.br/miscelanea/murphy.htm}{Lei de Murphy}}.%

Por fim, lembre-se que é normal ficar nervoso perante uma platéia, e não há uma
cura genérica para este problema. Há muitas sugestões de como lidar com isso\footnote{\url{http://www.wikihow.com/Overcome-Stage-Fright}},
inclusive uma que diz que o problema é você\footnote{\url{http://seriouspony.com/blog/2013/10/4/presentation-skills-considered-harmful}}.
Tente descobrir o que funciona melhor para si (boa sorte!).

\newcommand{\beamer}{{\sc{BEAMER}}}%
\section{\beamer}
A classe \beamer, disponível no \acrshort{CTAN}\footnote{\url{http://www.ctan.org/pkg/beamer}},
é a recomendada para criar apresentações. Não só possibilita um resultado visualmente
interessante, como também aproveita parte do texto escrito em \LaTeX. O manual%
\footnote{\url{http://www.ctan.org/tex-archive/macros/latex/contrib/beamer/doc/beameruserguide.pdf}}
oferece instruções sobre o uso da classe e, principalmente, diretrizes para criar
apresentações (especialmente as Seções 4 e 5 do Capítulo I).