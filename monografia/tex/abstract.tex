The University of Brasília (UnB) suffers from a big problem of student drop-out, which has
academics, economics and social consequences. UnB's approach to the problem consist
of separating it's students in two groups: those that are in risk of dropping out
and those that aren't and counsel the students in the former group. This research
goal is to enhance UnB's approach, by developing a predictive system capable of
indicating which students have higher risks of dropping out. The CRISP-DM model is
adopted in this research. 
This document registers what has been done in the initial part of the research, and
propose a schedule for the final part. In the initial part, the goal established was
to complete the business
understanding, data understanding and data preparation phases of the CRISP-DM. 
The schedule for the final part includes the modeling and the evaluation
phases.  
Despite the fact that we were not able to calculate some attributes, the initial
part of the research was made according to planned and the schedule presented
demonstrate the feasibility of finishing the end part of the research in the next
semester. 
