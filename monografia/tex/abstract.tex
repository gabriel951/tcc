% problem definition and current approach
The University of Brasília (UnB) suffers from a big problem of student drop out, which has
academic, economic and social consequences. UnB's approach to the problem consist
of separating it's students in two groups: those that are in risk of dropping out
and those that aren't and counsel the students in the former group. 
% my proposal
The goal of the present research is the development of a predictive system
capable of indicating the risk of a student dropping out. This way,
UnB could act before it's too late and also act according to the risk presented by a
student. 
% methodology
For the development of the predictive system, data of undergraduate students from
computer science related courses that entered and left UnB from 2000 to 2016 was
used. The data does not contain student identification. Machine learning models were
trained and their performance was analysed on test dataset. Machine learning models
studied were Naive Bayes, ANN, SVR, Linear Regressor and Random Forests).
% results/conclusion
Machine learning models got, in general, good performance. The best performance came
from the linear regressor. Results obtained indicate potential in using machine
learning to predict the risk of students dropping out of university. 
