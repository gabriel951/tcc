% problem definition and current approach
The University of Brasília (UnB) suffers from a problem of student drop out, which has
academic, economic and social negative consequences. UnB's approach to the problem
consist of separating it's students in two groups: those that are in risk of dropping
out and those that are not, and counsel the students in the former group. 
% my proposal
The goal of the present research is the development of a predictive system
capable of indicating the risk of a student dropping out. This way,
UnB could act before it's became late and also act according to the risk presented by a
student. 
% methodology
For the development of the predictive system, data of undergraduate students from
computer science related courses that entered and left UnB from 2000 to 2016 were
used. The data do not contain student identification. Machine learning algorithms were
used and their performance was evaluated. Algorithms included were Naive Bayes, ANN,
SVR, Linear Regressor and Random Forests).
% results/conclusion
Machine learning algorithms got, in general, good performance. The best performance came
from the linear regressor. Results obtained indicate potential in using machine
learning to predict the risk of students dropping out of university for the courses
related to computer science. Because the methodology did not use any concept from
this area of knowledge, this approach can be used for other courses. 
