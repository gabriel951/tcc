As etapas realizadas, seguindo o modelo de processos CRISP-DM, permitiram o
entendimento do negócio (por meio da definição do problema e da proposta de solução),
o entendimento dos dados e a preparação dos dados (feita parcialmente, já que alguns
atributos não puderam ser calculados).  
Foi possível definir quais são os atributos importantes para a predição almejada
e pôde-se observar a distribuição de cada um deles (por meio dos histogramas) e
a correlação entre eles (por meio do coeficiente de correlação de Kendall). Assim
sendo, conclui-se que o objetivo traçado para o trabalho nesse início de pesquisa foi
cumprido. Por fim, o calendário traçado para o próximo semestre é viável, indicando
portanto que a pesquisa deverá ser concluída no primeiro semestre.  
