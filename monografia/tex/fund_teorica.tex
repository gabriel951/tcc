Neste capítulo, descreve-se a fundamentação teórica necessária para compreender a
pesquisa. Assim, nas seções seguintes explica-se a problemática da evasão nas universidades
públicas, as especifidades da \acrshort{UnB}, o modelo de processos CRISP-DM, modelos
de aprendizagem de máquina adotados e as medidas de desempenho para tais modelos. 

\section{Evasão nas Universidades} 
A evasão nas universidades nacionais é um problema que traz desperdícios acadêmicos,
sociais e econômicos, tendo sida estudada por diversos autores. Apesar disso, o
conceito de evasão pode variar  de acordo com a pesquisa,
já que pode-se considerar evasão do curso, evasão da instituição ou evasão do
sistema educacional \cite{mec_conceito_evasao}. 

\section{Especificidades da UnB}
A UnB possui algumas particularidades importantes tanto em relação às notas dos
alunos quanto aos procedimentos adotados para alunos em risco de evadir. Tais
especificidades são apresentadas a seguir. 

\subsection{Procedimentos da UnB Para Alunos com Risco de Evadir}
\par Os critérios para que um aluno seja desligado na UnB são mostrados na Figura
\ref{desligamento}. 
\begin{figure}[!ht]
    \caption{Critérios Para Desligamento na UnB}
    \centering
    \includegraphics[width = 18cm]{desligamento.png}
    \label{desligamento}
\end{figure}
 
A abordagem da UnB para evitar desligamento consiste de separar os alunos em dois
grupos (alunos em condição e alunos que não estão em condição) e ter os alunos em
condição orientados. Os critérios para que um aluno esteja em condição são
\cite{manual_calouro}: 
\begin{itemize}
    \item Ter duas reprovações na mesma disciplina obrigatória
    \item Não ser aprovado em quatro disciplinas do curso em dois períodos regulares
        consecutivos
    \item Chegar ao último período letivo do curso sem a possibilidade de concluir
\end{itemize}

\subsection{O Índice de Rendimento Acadêmico}
Outra particularidade da UnB diz respeito às notas que seus alunos recebem. Para cada
matéria, os alunos podem receber as menções disponíveis na Tabela \ref{mencao_unb} 
\cite{manual_calouro}: 

\begin{table}
\begin{center}
\begin{tabular}[c]{|c|c|}
    \hline
    \textbf{Menção} & \textbf{Nota de 0 a 10} \\
    \hline
    SS & 9,0 a 10 \\
    \hline
    MS & 7,0 a 8,9 \\
    \hline
    MM & 5,0 a 6,9 \\
    \hline
    MI & 3,0 a 4,9 \\
    \hline
    II & 0,1 a 2,9 \\
    \hline
    SR & 0 \\
    \hline
\end{tabular}
\end{center}
\caption{Equivalência entre Menção Obtida por um Aluno e Nota Final na Disciplina}
\label{mencao_unb}
\end{table}

Cada aluno tem também um índice de rendimento acadêmico (\acrshort{IRA}), um valor entre 0 e 5
que sumariza as menções obtidas por um aluno, desde que entrou na UnB. Segundo
\cite{manual_calouro}, O IRA é calculado da seguinte forma:  
\begin{equation}
    IRA = (1 - \frac{0,6 * DT_b + 0,4 * DT_p}{DC}) * 
    (\frac{\sum_{i}P_i * CR_i * Pe_i}{\sum_{i} CR_i * Pe_i})
\end{equation}

em que os símbolos da fórmula são explicados a seguir: 
\begin{itemize}
    \item $DT_b$: Número de disciplinas obrigatórias trancadas
    \item $DT_p$: Número de disciplinas optativas trancadas
    \item $DC$: Número de disciplinas matriculadas (incluindo as trancadas)
    \item $P_i$: Peso da menção, onde (SS = 5, MS = 4, MM = 3, MI = 2, II = 1, SR =
        0)
    \item $Pe_i$: Período em que uma dada disciplina foi cursada, obedecendo à
        limitação min(6, Período). 
    \item $CR_i$: Número de Créditos de uma disciplina
\end{itemize}

\section{CRISP-DM}
O \acrshort{CRISP-DM}(do inglês \textit{CRoss-Industry Standard Process for Data Mining}) é um
modelo de processos para mineração de dados que se divide em 6 fases principais
\cite{crispdm}:
\begin{enumerate}
    \item Entendimento do negócio
    \item Entendimento dos dados
    \item Preparação dos Dados
    \item Modelagem
    \item Avaliação
    \item Implantação
\end{enumerate}
Cada uma dessas fases é explicada a seguir.

\subsection{Entendimento do Negócio}
Nesta etapa determinam-se os objetivos do negócio (através do entendimento do que o cliente
almeja), avalia-se a situação atual, determinam-se os objetivos da mineração de dados
e produz-se um planejamento do projeto \cite{crispdm}. Uma importante tarefa dentro
dessa fase do CRISP-DM é o levantamento do estado da arte. 

\subsection{Entendimento dos Dados}
Essa fase é composta pelas tarefas de: coleta inicial dos dados, descrição dos dados,
exploração dos dados e verificação da qualidade dos dados \cite{crispdm}.
Na fase de entendimento dos dados, juntamente com a fase de preparação dos dados, 
é comum a utilização de estatística descritiva. A
estatística descritiva permite identificar problemas de qualidade nos dados, como por
exemplo a identificação de valores de atributos faltantes ou a identificação de
\textit{outliers} ou a identificação de atributos com cardinalidade irregular
\cite{ml_book}. Técnicas comuns de estatística descritiva incluem histogramas,
gráficos de barra e \textit{boxplots}. 

\subsection{Preparação dos Dados}
Essa fase é composta pelas tarefas de: seleção dos dados, limpeza dos dados, construção
dos dados, integração dos dados e formatação dos dados \cite{crispdm}. Para a tarefa
de seleção dos dados, é útil ter uma noção do grau de correlação entre as variáveis. 
Caso a correlação entre algum par de variável seja muito alta, a eliminação de uma
delas pode acarretar em um modelo mais simples de ser compreendido (sem prejudicar o
desempenho).
\par Assim, pode-se usar um coeficiente de correlação para medir o grau de dependência
entre atributos e decidir pela eventual eliminação de algum. Existem vários testes
para se medir a correlação entre variáveis, como o coeficiente de Spearman e o
coeficiente de correlação de Kendall \cite{kendall}. 

\subsection{Modelagem} 
Essa fase é composta pelas tarefas de: seleção da técnica de modelagem, geração dos
casos de teste, construção do modelo e avaliação do modelo \cite{crispdm}. É nessa
fase que utilizam-se os algoritmos de aprendizagem de máquina como árvores de decisão,
redes neurais e SVM's \cite{ml_second_book}. 

\subsection{Avaliação} 
Enquanto a tarefa de avaliação da fase de modelagem tem como objetivo lidar como
fatores como a precisão e a generalidade do modelo, essa fase tem como objetivo
averiguar o quanto o modelo cumpre com os objetivos de negócio \cite{crispdm}. São
tarefas dessa fase: avaliação de resultados, revisão de processos e decisão de quais
serão os próximos passos. 

\subsection{Implantação} 
A fase final do CRISP-DM é a implantação. Nessa fase, são realizadas as tarefas de:
planejar a implantação, planejar monitoramento e manutenção, produzir relatório final
e revisar o projeto \cite{crispdm}.

\section{Técnicas de Aprendizagem de Máquina}
Nesta seção fazem-se breve explanações sobre as técnicas de aprendizagem de máquina
usadas na pesquisa. Explica-se sobre as árvores de decisão e \textit{random forests},
regressor linear, SVR's, redes neurais, \textit{Naive Bayes} e o ZeroR. 

% árvores de decisão, random forest, regressor linear, redes neurais ...
\subsection{Árvores de Decisão e Random Forest}
Árvores de Decisão são uma técnica de aprendizagem de máquina bastante usada em
pesquisas científicas, em uma variedade de contextos \cite{ml_mitchell}. O modelo
adota uma representação em árvore, com cada vértice fazendo referência à um teste a ser
feito para um atributo de uma instância e cada uma das arestas indicando um dos
possíveis valores do atributo. A Figura \ref{dec_tree}, exemplo encontrado em
\cite{ml_mitchell} mostra uma árvore de decisão para o problema de decidir se uma
manhã de sol está adequada para a prática de jogar tênis. Note que as folhas das
árvores correspondem à decisão à ser tomada. 

\begin{figure}[!ht]
    \caption{Árvore de Decisão de Modo a Escolher se uma Manhã está Adequada à
    Prática de Tênis, com Base nas Condições Climáticas.} 
    \centering
    \includegraphics[width = 8cm]{dec_tree.png}
    \label{dec_tree}
\end{figure}

\par Por fim, \textit{Random Forests} é um método de aprendizagem de máquina (baseado nas
árvores de decisão) adequado para as tarefas
de classificação e regressão. \textit{Random Forests} operam construindo várias árvores de
decisão durante o treino e fornecendo como saída a predição média (no caso de
regressão) ou a moda das classes (no caso de classificação) \cite{random_forest}. 

\subsection{Regressão Linear}
O regressor linear é um modelo linear que tenta prever a relação entre uma variável
dependente $y$ em termos de uma ou mais variáveis independentes, denotadas por $X$.
Se apenas uma variável independente é usada, a regressão linear é dita simples,
enquanto que se mais de uma variável é utilizado, a regressão linear é dita
múltipla. 
\par Por fim, pode-se tentar prever múltiplas variáveis dependentes
correlacionadas, caso no qual o modelo é dito regressor linear multivariado. Nessa
pesquisa, fez-se regressão linear múltipla multivariada. 
De modo geral, modelos lineares costumam não ser muito propensos à
\textit{overfitting} e são boas alternativas iniciais para problemas de aprendizagem
de máquina \cite{ml_book}. 

\subsection{SVR}
SVM (do inglês \textit{Support Vector Machine}) é um classificador baseado em um hiperplano
separador. Assim, após treinar, tal modelo fornece um hiperplano otimizado e, com
base nele,
categoriza novos exemplos. Por meio de seu \textit{kernel}, SVM's conseguem realizar
transformações não lineares de alta dimensão \cite{ml_second_book}. Tal modelo de
aprendizagem de máquina é fácil de usar e tem bom desempenho na prática, o que
explica assim sua popularidade \cite{ml_second_book}. O método utilizado na SVM pode ser
extendido para regressão. Nesse caso, o método é chamado SVR (do inglês
\textit{Support Vector Machine for Regression}). 

\subsection{ANN}
ANN's (do inglês \textit{Artificial Neural Networks}) ou Redes Neurais são um modelo
de aprendizagem de máquina inspirado na biologia que tem bastante sucesso em
aplicações de áreas diversas, indo de visão computacional
\cite{ml_second_book} à estratégia de jogos como Go \cite{google_ann}. Entretanto, como
redes neurais são um modelo muito flexível, com grande poder de aproximação, é fácil
incorrer em \textit{overfitting} ao se escolher tal modelo \cite{ml_second_book}. 
\par ANN's são um conjunto de unidades conectadas, chamadas de neurônios (em
analogia aos neurônios do cérebro humano). Se dois neurônios estão conectados (uma
conexão é dita sinapse, outra analogia ao cérebro humano), um
deles pode transmitir um sinal ao outro. As sinapses tem um peso, que varia conforme
ocorre o aprendizado, que determina a intensidade na qual um sinal é propagado. 
\par Os neurônios tipicamente são organizados em camadas, as quais realizam
diferentes transformações nas entradas que recebem. Costuma-se diferenciar entre a
camada de entrada, a camada escondida e a camada de saída, conforme pode ser visto na
Figura \ref{ann_layers}. Assim, o funcionamento de uma rede neural ocorre, de modo bem
superficial, da seguinte forma: uma nova instância é fornecida à rede
neural (na camada de entrada), ativam-se determinados neurônios da rede neural (de
acordo com os dados de entrada) e um determinado valor é fornecido na camada de
saída, de acordo com os neurônios que foram ativados.  

\begin{figure}[!ht]
    \caption{Esquema Simplificado das Camadas de uma Rede Neural, em que São
    Mostradas a Camada de Entrada, a Camada Escondida e a Camada de Saída}
    \centering
    \includegraphics[width = 8cm]{ann_layers.png}
    \label{ann_layers}
\end{figure}

\subsection{Naive Bayes}
O \textit{Naive Bayes} é um classificador
probabilístico baseado na aplicação do teorema de Bayes com a premissa de
independência entre os atributos \cite{ml_book}. Um parâmetro importante para tal
modelo é o tipo de distribuição que se assume que os atributos possuem. Alguns
exemplos de distribuições são: Gaussiana, multinomial e Bernoulli. 

\subsection{ZeroR}
O ZeroR é um dos métodos de classificação mais simples. Tal modelo ignora os
atributos e sempre prevê a classe majoritária. Embora não tenha um grande poder de
predição, o ZeroR é útil para determinar um desempenho básico, que os modelos de
aprendizagem de máquina devem ser capazes de superar. 

\section{Métricas Para Avaliação de Desempenho em Aprendizagem de Máquina}
Existem várias métricas para a avaliação do desempenho de modelos de aprendizagem de
máquina. Dentre as técnicas comuns para o problema de classificação supervisionado,
pode-se citar a precisão, o \textit{recall} e a \textit{F-measure}.  

\par A precisão é definida como sendo a razão entre a quantidade de verdadeiros
positivos ($TP$) e a quantidade de positivos ($TP + FP$). Assim sendo, a precisão
traduz o quão frequentemente o modelo acerta quando este faz uma previsão positiva.
A fórmula para o cálculo da precisão é mostrada a seguir: 
\begin{equation}
    Precisão = \frac{TP}{TP + FP}
\end{equation}

Já o \textit{recall} é definido como a razão entre a quantidade de verdadeiros
positivos pela soma de verdadeiros positivos com os falsos negativos. Desse modo,
\textit{recall} traduz o quão confiante podemos estar que todas as instâncias
positivas foram encontradas pelo modelo. A fórmula para o cálculo do \textit{recall} é
mostrada a seguir: 
\begin{equation}
    Recall = \frac{TP}{TP + FN}
\end{equation}

\par Precisão e \textit{recall} fornecem diferentes informações, ambas 
úteis. Desse modo, faz sentido definir uma métrica que leve em conta ambas as
informações para avaliar o desempenho. A \textit{F-measure} é essa métrica, sendo
definida como a média harmônica entre a precisão e o \textit{recall}. Precisão,
\textit{recall} e \textit{F-measure} assumem valores de 0 a 1.
