% TODO: falar do modelo crips-dm, se quiser encher linguiça
Neste capítulo, descreve-se a fundamentação teórica necessária para compreender a
pesquisa. Assim, nas seções seguintes explica-se a problemática da evasão nas universidades
públicas, o procedimento da UnB para evitar tal situação e o modelo de processos
CRISP-DM. 

\section{Evasão nas Universidades} 
A evasão nas universidades nacionais é um problema que traz desperdícios acadêmicos,
sociais e econômicos, tendo sida estudada por diversos autores. Apesar disso, o
conceito de evasão pode variar \cite{mec_conceito_evasao} de acordo com a pesquisa,
já que pode-se considerar evasão do curso, evasão da instituição ou evasão do
sistema educacional. 

\section{Procedimentos da UnB Para Alunos com Risco de Evadir}
\par Os critérios para que um aluno seja desligado na UnB são mostrados na Figura
\ref{desligamento}. 
\begin{figure}[!ht]
    \caption{Critérios Para Desligamento na UnB}
    \centering
    \includegraphics[width = 18cm]{desligamento.png}
    \label{desligamento}
\end{figure}
 
A abordagem da UnB para evitar desligamento consiste de separar os alunos em dois
grupos (alunos em condição e alunos que não estão em condição) e ter os alunos em
condição orientados. Os critérios para que um aluno esteja em condição são
\cite{manual_calouro}: 
\begin{itemize}
    \item Ter duas reprovações na mesma disciplina obrigatória
    \item Não ser aprovado em quatro disciplinas do curso em dois períodos regulares
        consecutivos
    \item Chegar ao último período letivo do curso sem a possibilidade de concluir
\end{itemize}

% TODO: explicar como é calculado o IRA
\section{O Índice de Rendimento Acadêmico}


\section{CRISP-DM}
O CRISP-DM (do inglês \textit{CRoss-Industry Standard Process for Data Mining}) é um
modelo de processos para mineração de dados que se divide em 6 fases principais
\cite{crispdm}:
\begin{enumerate}
    \item Entendimento do negógio
    \item Entendimento dos dados
    \item Preparação dos Dados
    \item Modelagem
    \item Avaliação
    \item Implantação
\end{enumerate}
Cada uma dessas fases é explicada a seguir.

\subsection{Entendimento do Negócio}
Nesta etapa determinam-se os objetivos do negócio (entendimento do que o cliente
almeja), avalia-se a situação atual, determinam-se os objetivos da mineração de dados
e produz-se um planejamento do projeto \cite{crispdm}. Uma importante tarefa dentro
dessa fase do CRISP-DM é o levantamento do estado da arte. 

\subsection{Entendimento dos Dados}
Essa fase é composta pelas tarefas de: coleta inicial dos dados, descrição dos dados,
exploração dos dados e verificação da qualidade dos dados \cite{crispdm}.
Na fase de entendimento dos dados, juntamente com a fase de preparação dos dados, 
é comum a utilização de estatística descritiva. A
estatística descritiva permite identificar problemas de qualidade nos dados, como por
exemplo a identificação de valores de atributos faltantes ou a identificação de
\textit{outliers} ou a identificação de atributos com cardinalidade irregular
\cite{ml_book}. Técnicas comuns de estatística descritiva incluem histogramas,
gráficos de barra e \textit{boxplots}. 

\subsection{Preparação dos Dados}
Essa fase é composta pelas tarefas de: seleção dos dados, limpeza dos dados, construção
dos dados, integração dos dados e formatação dos dados \cite{crispdm}. Para a tarefa
de seleção dos dados, é útil ter uma noção do grau de correlação entre as variáveis. 
Caso a correlação entre algum par de variável seja muito alta, a eliminação de uma
delas pode acarretar em um modelo mais simples de ser compreendido (sem prejudicar o
desempenho).
\par Assim, pode-se usar um coeficiente de correlação para medir o grau de dependência
entre atributos e decidir pela eventual eliminação de algum. Existem vários testes
para se medir a correlação entre variáveis, como o coeficiente de Spearman e o
coeficiente de correlação de Kendall \cite{kendall}. 
%TODO: escrever mais um pouco aqui caso deseje, sobre Kendall

\subsection{Modelagem} 
Essa fase é composta pelas tarefas de: seleção da técnica de modelagem, geração dos
casos de teste, construção do modelo e avaliação do modelo \cite{crispdm}. É nessa
fase que utilizam-se os algoritmos de aprendizagem de máquina como árvores de decisão,
redes neurais e SVM's \cite{ml_second_book}. 

\subsection{Avaliação} 
Enquanto a tarefa de avaliação da fase de modelagem tem como objetivo lidar como
fatores como a precisão e a generalidade do modelo, essa fase tem como objetivo
averiguar o quanto o modelo cumpre com os objetivos de negócio \cite{crispdm}. São
tarefas dessa fase: avaliação de resultados, revisão de processos e decisão de quais
serão os próximos passos. 

\subsection{Implantação} 
A fase final do CRISP-DM é a implantação. Nessa fase, são realizadas as tarefas de:
planejar a implantação, planejar monitoramento e manutenção, produzir relatório final
e revisar o projeto \cite{crispdm}.

% TODO: incluo mais sobre a modelagem?
