% TODO: problema referenciar artigos unpublished?
Neste capítulo, faz-se uma introdução do trabalho. A definição do problema é feita e
a proposta de solução é apresentada. Em seguida, descrevem-se os objetivos traçados e
explica-se a estrutura geral do documento em questão. 

\section{Definição do Problema}
% fala do problema
A evasão de alunos de graduação nas universidades brasileiras já foi estudada por
diversos autores, tratando-se de um problema que traz desperdícios acadêmicos,
sociais e econômicos.
A Universidade de Brasília (UnB) não é exceção, sendo afetada
significativamente pelo problema \footnote{A UnB teve um prejuízo com evasão estimado
em 95,6 milhões, segundo o Correio Brasiliense na reportagem do dia 10/10/2015 ``Evasões na
Universidade de Brasília causam prejuízo de R\$ 95mi''\cite{correio}}.

% o q a unb faz
\par Atualmente a abordagem da UnB consiste de separar os seus alunos entre aqueles
que estão em risco de serem desligados e precisam cumprir condição para evitar o
desligamento (tais alunos são
ditos alunos em condição) e aqueles que não estão, e ter os alunos em condição
supervisionados por um orientador \cite{manual_calouro}. 
Essa abordagem, entretanto, apresenta problemas:
    \begin{itemize}
        \item Os alunos (uma amostra muito diversificada) são classificados em apenas
    dois grupos.
        \item A UnB age apenas quando o aluno já se encontra em condição, quando pode
    já ser tarde demais.
    \end{itemize}

\section{Proposta de Solução}
    % minha solução
    A pesquisa aqui descrita justifica-se como uma tentativa de melhorar a abordagem
    atual da UnB para o problema da evasão. Propõe-se utilizar dados passados para
    a criação de um sistema previsor que seja capaz de identificar alunos em risco de
    serem desligados. O sistema previsor fornece 3 valores positivos $v_1$, $v_2$,
    $v_3$ que somam 1 e indicam respectivamente a chance do aluno se graduar, ser
    desligado (evadir) e migrar de curso. 
    \par Valores mais próximos de 0 indicam um baixo risco do evento em
    questão acontecer, enquanto valores mais próximos de 1 indicam um alto risco. Por
    exemplo, um valor de $v_1$ próximo de 1 indica que o aluno tem grande chance de se
    graduar, ao passo que um valor de $v_3$ próximo de 0 indica que o aluno não tem 
    muito risco de migrar de curso. 
    \par Os dados utilizados são dados descaracterizados de alunos de
    graduação da UnB, contendo tanto informações de perfil quanto a forma de ingresso e
    as menções nas matérias da UnB. 
    \par Potencialmente, o sistema previsor permitirá a
    UnB agir antes de um aluno entrar em condição. Outra vantagem será a possibilidade
    de agir de acordo com o risco de evasão apresentado por cada
    aluno. Ou seja, a proposta de solução permitirá agir com antecedência e
    flexibilidade. 

\section{Objetivos}
O objetivo da pesquisa é investigar a aplicação de técnicas de mineração de dados
para o desenvolvimento de modelos de predição da conclusão do curso por alunos de
graduação da UnB. Os objetivos específicos são: 
\begin{itemize}
    \item Avaliar a viabilidade dessa abordagem para
          uma área mais restrita do conhecimento para, em uma segunda fase, aplicar a
          metodologia desenvolvida para os demais cursos de graduação ofertadas pela UnB.
    \item Induzir modelos que indiquem a possibilidade do aluno concluir, evadir ou
          migrar de curso. 
\end{itemize}  

Na presente etapa dessa pesquisa é investigada apenas a indução de modelos preditores
para os cursos da área de computação ofertados pela UnB, ou seja: Ciência da
Computação, Engenharia de Computação, Engenharia de Controle e Automação (também
chamada Engenharia Mecatrônica), Engenharia de Software, Engenharia de Redes e
Licenciatura em Computação. 

\section{Organização do Restante do Documento}
% organização do documento
\par Descreve-se a seguir a organização do restante do documento. 
No Capítulo 2 explica-se a fundamentação teórica necessária para a compreensão da
pesquisa em questão. Já no Capítulo 3, descreve-se a metodologia
adotada na pesquisa. No Capítulo 4 apresentam-se e discutem-se os resultados
obtidos. Finalmente, no Capítulo 5, apresenta-se a conclusão e apontam-se sugestões para
trabalhos futuros. Apresenta-se a parte de estatística descritiva separadamente no
apêndice A. 
