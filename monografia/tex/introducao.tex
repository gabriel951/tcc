% TODO: problema referenciar artigos unpublished?
Neste capítulo, faz-se uma introdução do trabalho. A definição do problema é feita e
a proposta de solução é apresentada. Em seguida, descrevem-se os objetivos traçados e
explica-se a estrutura geral do documento em questão. 

\section{Definição do Problema}
% fala do problema
A evasão nas universidades brasileiras já foi estudada por diversos autores,
tratando-se de um problema que traz desperdícios acadêmicos, sociais e econômicos.
A UnB (Universidade de Brasília) não é exceção, sendo afetada significativamente pelo
problema \footnote{A UnB teve um prejuízo com evasão estimado em 96,5 milhões,
segundo o Correio Brasiliense \cite{correio}}.

% o q a unb faz
\par Atualmente a abordagem da UnB consiste de separar os seus alunos entre aqueles
que estão em condição e aqueles que não estão e ter os alunos em condição orientados
    \cite{manual_calouro}. Essa abordagem, entretanto, apresenta problemas:
    \begin{itemize}
        \item Os alunos (uma amostra muito diversificada) são classificados em apenas
    dois grupos.
        \item A UnB age apenas quando o aluno já se encontra em condição, quando pode
    já ser tarde demais.
    \end{itemize}

\section{Proposta de Solução}
    % minha solução
    A pesquisa aqui descrita justifica-se como uma tentativa de melhorar a abordagem
    atual da UnB para o problema da evasão. Propõe-se utilizar dados passados para
    a criação de um sistema previsor que seja capaz de identificar alunos em risco de
    serem desligados. O sistema previsor fornece 3 valores $v_1$, $v_2$, $v_3$ entre 0
    e 1, que indicam respectivamente a chance do aluno se graduar, ser desligado (evadir)
    e migrar de curso. 
    \par Valores mais próximos de 0 indicam um baixo risco do evento em
    questão acontecer, enquanto valores mais próximos de 1 indicam um alto risco. Por
    exemplo, um valor de $v_1$ próximo de 1 indica que o aluno tem grande chance de se
    graduar, ao passo que um valor de $v_3$ próximo de 0 indica que o aluno não tem 
    muito risco de migrar de curso. Para isso, diferentes modelos de aprendizagem de
    máquina serão estudados. 
    \par Os dados utilizados são dados descaracterizados de alunos de
    graduação da UnB, contendo tanto informações de perfil quanto a forma de ingresso e
    as menções nas matérias da UnB. 
    \par Caso bem sucedido, o sistema previsor permitirá a
    UnB agir antes de um aluno entrar em condição. Outra vantagem será a possibilidade
    de agir de acordo com o risco de evasão apresentado por cada
    aluno. Ou seja, a proposta de solução permitirá agir com antecedência e
    flexibilidade. 

\section{Objetivos}
    São dois os objetivos deste documento: 
    \begin{itemize}
    \item Explicar as etapas realizadas para se desenvolver o sistema previsor de
        risco de evasão, justificando as escolhas realizadas.
    \item Discutir o desempenho deste sistema, analisando assim sua viabilidade e
        apontando caminhos futuros de pesquisa nessa linha.
\end{itemize}

\section{Organização do Restante do Documento}
% organização do documento
\par Descreve-se a seguir a organização do restante do documento. 
No capítulo 2 explica-se a fundamentação teórica necessária para a compreensão da
pesquisa em questão. Já no capítulo 3, descreve-se a metodologia
adotada na pesquisa. No capítulo 4 apresentam-se e discutem-se os resultados
obtidos. Finalmente, no capítulo 5, apresenta-se a conclusão e apontam-se sugestões para
trabalhos futuros. Apresenta-se a parte de estatística descritiva separadamente no
apêndice A. 
