Neste capítulo, descreve-se a metodologia usada em toda a pesquisa. 
Detalha-se como foi feito o levantamento
do estado da arte, a obtenção e utilização dos dados, a seleção de
atributos, a eliminação de \textit{outliers}, a análise preliminar por meio de estatística
descritiva, a escolha dos modelos e a forma de avaliar o desempenho destes.  

\section{Levantamento do Estado da Arte}
Foi feito o levantamento do estado da arte através da leitura de diversos artigos, de
modo a entender quais fatores são importantes para a evasão \cite{adeodato}
\cite{hoed_fatores} \cite{dropout_finland}, como técnicas de
aprendizagem de máquina podem ser utilizadas para resolver o problema \cite{adeodato}
 \cite{data_mining_retention} 
e como trabalhar especificamente com os dados da UnB \cite{manual_calouro} 
\cite{hoed_sobrevivencia}. 

\section{Obtenção e Utilização dos Dados}
Obtiveram-se informações descaracterizadas relativas aos dados sociais e ao
desempenho acadêmico de alunos da UnB. Todos os dados utilizados vieram de uma só
fonte, de modo que o comum problema encontrado na área de mineração de dados de
garantir a consistência dos dados entre várias fontes não foi enfrentado. 
\par Optou-se por restringir a pesquisa apenas aos alunos que entraram a partir de
2000 e saíram até 2016, de modo a trabalhar com dados mais recentes. 
Para simplificar a análise, decidiu-se trabalhar apenas com uma área
específica (computação), de modo que apenas os seguintes cursos foram considerados: 
\begin{itemize}
    \item Ciência da Computação (Bacharelado)
    \item Computação (Licenciatura)
    \item Engenharia da Computação 
    \item Engenharia de Redes
    \item Engenharia de Software
    \item Engenharia Mecatrônica
\end{itemize}
\par Decidiu-se que os dados de alunos que tivessem entrado entre 2000 e 2010 seriam
usados como dados de treino, enquanto que os dados de alunos que tivessem entrado de
2010 até 2016 seriam usados como dados de teste.

\section{Seleção Preliminar de Atributos}
Com base no levantamento do estado da arte feito, selecionou-se quais atributos
teriam melhor condição de serem significativos para que um aluno fosse ou não
desligado. Assim sendo, lista-se a seguir os atributos sociais considerados em uma análise
inicial:
\begin{itemize}
        \item Cotista ou não
        \item Curso
        \item Forma de Ingresso
        \item Idade
        \item Raça
        \item Sexo
        \item Tipo da Escola 
\end{itemize}

Além de dados sociais, utilizaram-se os seguintes atributos (relativos ao desempenho
acadêmico): 
\begin{itemize}
    \item Coeficiente de Melhora Acadêmica
    \item Indicador de Aluno em Condição
    \item Média do Período
    \item Posição em relação ao semestre que ingressou
    \item Quantidade de créditos já integralizados
    \item Taxa de Aprovação, Taxa de Reprovação e Taxa de Trancamento
    \item Taxa de aprovação na disciplina mais difícil de cada semestre
\end{itemize}

\par O coeficiente de melhora acadêmica é definido como sendo a razão entre o IRA do
aluno em um semestre pelo IRA do aluno no semestre anterior. Dessa maneira, o
coeficiente de melhora acadêmica mostra se o desempenho do aluno está
melhorando, piorando ou se encontra estável.  
\par A média do período é calculada da mesma forma que o IRA, mas considerando apenas
as menções obtidas em um determinado semestre. Assim como o IRA, a média do período
varia entre 0 e 5.
\par A posição em relação ao semestre $P$ para um determinado aluno é definida como sendo:
o número de alunos com IRA maior que o do estudante em questão
(considerando apenas aqueles que entraram no mesmo curso do estudante, no mesmo ano e
no mesmo semestre). Assim, um aluno com posição $P = 0$ é aquele que tem o maior
IRA em relação a seus colegas que entraram no mesmo curso durante o mesmo ano e
semeste.  
\par Como todos os cursos da UnB requerem uma quantidade de créditos mínima para
graduação, incluiu-se o atributos quantidade de créditos já integralizados. 
\par A taxa de aprovação é definida como a razão entre o número de matérias cursadas
pelo aluno com aprovação pelo número de matérias cursadas pelo aluno. Analogamente, a
taxa de reprovação é definida como a razão entre o número de matérias cursadas pelo
aluno com reprovação pelo número de matérias cursadas pelo aluno. Deve-se dizer
incluir a taxa de aprovação e a de reprovação não é, a priori, redundante, já que
além de ser aprovado ou reprovado em uma matéria, outra possibilidade é o aluno
realizar o trancamento.  Pensando nisso, definiu-se a taxa de trancamento como sendo
a razão entre o número de matérias trancadas pelo aluno pelo o número de matérias
cursadas pelo aluno.
\par Por fim, a taxa de aprovação na disciplina mais difícil do semestre é definida
como sendo a razão entre o número de aprovações na disciplina mais difícil do
semestre pelo número de semestres na UnB. A disciplina considerada como a mais
difícil do semestre é aquela com a maior taxa de reprovação. 
\par Deve-se destacar que, para um mesmo aluno, os atributos relacionados ao
desempenho variam conforme o semestre considerado. O mesmo não ocorre para os
atributos sociais. 

\section{Eliminação de Atributos devido a Missing Values}
Optou-se por eliminar atributos cuja quantidade de entradas com \textit{missing
values} fosse superior à 40\%. Assim, eliminaram-se os atributos raça e tipo da
escola. Os gráficos de barra para tais atributos se encontram em
\ref{graf_miss_value}.

\section{Eliminação de Outliers}
Decidiu-se não trabalhar com casos de alunos que após ingressar na universidade não
demonstraram interesse em cursar matérias (por exemplo, aqueles que reprovaram em
todas as disciplinas com SR). Tais casos foram tratados como \textit{outliers}.
Após a análise individual de cada caso, os \textit{outliers} foram eliminados do espaço
amostral. Eliminaram-se 230 estudantes do espaço amostral dessa maneira, ficando-se
assim com 4539 amostras.

\section{Estatística Descritiva}
Foi feita uma análise preliminar por meio de estatística descritiva. As subseções
seguintes explicam cada um dos procedimentos adotados. apresentam a mudança na base
de dados realizada, a mudança nos valores de alguns atributos e os gráficos de barra
e histogramas finais. 

\subsection{Mudança na Base de Dados}
\par Foi possível observar que os atributos variavam significativamente de acordo com
o curso. Isso se deve ao fato de cada curso ter currículo diferente dos
demais, alguns cursos serem de instituições diferentes (FT ou IE, por exemplo) e os
cursos possuírem ``maturidade'' diferentes (devido à data início de cada curso ser
diferente). Outra observação preliminar possível foi a de que a proporção de alunos
com idade avançada que se forma é menor do que a de alunos mais jovens. Essas
observações levaram a partição da base de dados original em 4 bases de dados: 
\begin{itemize}
    \item Alunos Sêniors: contém todos os alunos com mais de 30 anos
    \item Alunos Jovens de Licenciatura: contém todos os alunos com 30 anos ou menos
        que cursam licenciatura em Computação. O curso de licenciatura tem a
        peculiaridade de ser o único noturno. 
    \item Alunos Jovens de Computação: contém todos os alunos com 30 anos ou menos
        que cursam Ciência da Computação, Engenharia da Computação ou Engenharia de
        Software. 
    \item Alunos Jovens de Tecnologia da Informação: contém todos os alunos com 30
        anos ou menos que cursam Engenharia de Redes ou Engenharia Mecatrônica. Tais
        cursos tem a peculiaridade de estarem associados à FT (Faculdade de
        Tecnologia), diferentemente de todos os demais.  
\end{itemize}
Devido à baixa quantidade de alunos sêniors, não se separou tal categoria em grupos.
A evidência fornecida pela estatística descritiva para essa separação
encontra-se na seção \ref{justificativa_4_base_dados}.

\subsection{Mudança nos Valores de Atributos}
\par Dois atributos tiveram seus alguns de seus valores agrupados em categorias para
facilitar o posterior tratamento dos dados. Tais atributos são a forma de entrada e a
forma de saída. A distribuição original de tais atributos pode ser vista na seção 
\ref{just_mud_atr}. 
\par No atributo forma de entrada, de modo a eliminar os vários valores com poucas
instâncias na base de dados, criou-se a categoria ``outros''. Assim, apenas três 
categorias foram consideradas: vestibular, PAS e outros. 
\par Decidiu-se trabalhar com apenas três valores de forma de saída: graduou, evadiu
e migrou. Essa última categoria foi criada para agrupar os casos de transferência,
mudança de curso, mudança de turno e realização de um novo vestibular. 

\subsection{Gráficos de Barra e Histogramas}
Após as modificações descritas nas subseções anteriores, fez-se o gráfico de barra
para os atributos discretos e os histogramas para os atributos contínuos.
Considerou-se, para isso, a divisão dos dados nas 4 bases de dados descritas. Tais
resultados encontram-se na seção \ref{graf_bar_hist}.

\section{Eliminação de Atributos Relacionados ou Irrelevantes}
\par Na área de aprendizagem de máquina, uma etapa fundamental é a da engenharia de
atributos \cite{useful_ml}, em que garante-se que os modelos irão receber os
atributos certos, para que sejam capazes de aprender. Assim sendo, deve-se realizar a
eliminação de atributos que estejam muito relacionados entre si e também a eliminação
de atributos irrelevantes. 
 
\par Avaliou-se o grau de relacionamento entre os diversos atributos
que compunham o modelo por meio do coeficiente de correlação de Kendall. Optou-se por
eliminar uma variável de cada par que apresentasse mais de 80\% de correlação. Os
atributos taxa de aprovação e taxa de reprovação apresentaram, para todas as bases
de dados, alto relacionamento. Assim sendo, eliminou-se o atributo taxa de reprovação
de análises posteriores. 

\par Para descobrir os atributos irrelevantes, dentro daqueles originalmente
pensados, utilizaram-se árvores de decisão: os atributos que não apareceram nas
árvores de decisão foram eliminados das fases posteriores. Esse procedimento foi
feito para cada uma das 4 bases de dados. Para os alunos jovens da licenciatura,
obteve-se que o atributo curso não era relevante, o que era esperado já que todos os alunos
desse grupo tem o mesmo curso). Para os alunos sêniors, obteve-se que os atributos
curso, cota e taxa de trancamento são irrelevantes. Para os outros grupos, nenhum
atributo foi classificado como irrelevante. 

\section{Modelagem}
Explica-se nesta seção o que foi feito durante a fase de modelagem. Inicia-se
explanando como foi feita a divisão entre treino e teste.Depois, aborda-se como o
desbalanço entre os dados de treino e os dados de teste foi corrigido por meio de
\textit{oversampling}.
Em seguida, descreve-se como foi usado validação para estimar os melhores parâmetros
de alguns modelos de aprendizagem de máquina. Finalmente, explica-se como avaliou-se
o desempenho das várias técnicas implantadas.  

\section{Divisão em Treino e Teste}
Como tradicionalmente ocorre no domínio de análise preditiva, houve a separação dos
dados entre dados de treino e dados de teste. Os dados abrangiam alunos que entraram
no período de 2000 até 2016 e já saíram da universidade. Aqueles alunos que entraram
antes de 2010 formaram o conjunto dos dados de treino, enquanto que os alunos que
entraram de 2010 em diante formaram o conjunto dos dados de teste. 

\section{Necessidade de Oversampling}
Constatou-se uma grande diferença na proporção de alunos que se formava quando se
comparava os dados de treino e os dados de teste. De fato, 53\% dos
alunos dos dados de treino se mostrou capaz de conclui o curso, enquanto que apenas
18\% dos alunos dos dados de teste.
\par Para entender a explicação para esse fenômeno, considere, por exemplo, os alunos
ingressantes na UnB no ano de 2014. Aqueles que irão se graduar, não o farão até 2016
e, portanto, não aparecerão em nossa amostra. Por outro lado, alguns dos que irão
evadir o farão até 2016, aparecendo assim em nossa amostra. Essa é a explicação para
o desbalanceamento da proporção de alunos que graduaram nos dados de teste. 
\par Para corrigir o problema, usou-se a técnica de \textit{oversampling}:
adicionou-se aos dados de teste cópias idênticas de instâncias já presentes nestes
dados, até que a proporção de graduandos no treino e no teste ficasse igual (no caso,
53\% capazes de se graduar). 

\section{Uso de Validação Para Estimativa de Parâmetros}

\section{Avaliação de Desempenho}
% TODO: modelo funciona para vários semestres. 
% f-measure
