% definição do problema e abordagem atual
A evasão de alunos de graduação na Universidade de Brasília (UnB) traz consequências
acadêmicas, sociais e econômicas negativas. A abordagem da UnB para resolução deste problema
consiste em separar os alunos em dois grupos: aqueles que estão em risco de serem
desligados e precisam cumprir condição para evitar o desligamento e
aqueles que não estão, e ter os alunos em condição supervisionados por um orientador. 
% proposta de solução
Pensando em melhorar tal abordagem, a pesquisa em questão objetiva a concepção de um
sistema previsor capaz de indicar quais alunos estão com maior risco de não
conseguirem formar.  Desse modo, o sistema permitiria a UnB agir antes de um aluno
entrar em condição e agir de acordo com o risco de evasão apresentado por cada aluno.
% metodologia
Para o desenvolvimento do sistema previsor, dados descaracterizados de alunos de
graduação de cursos da área de computação que
ingressaram de 2000 até 2016 e já saíram da universidade foram utilizados.
Algoritmos de aprendizagem de máquina foram usados (\textit{Naive Bayes}, ANN, SVR,
Regressor Linear e Random Forests) para induzir modelos que tiveram seu desempenho
analisado. 
% resultados/conclusão
Os modelos obtidos testados tiveram, em geral, bom desempenho. O
melhor desempenho foi para modelos induzidos com regressão linear. Os resultados
obtidos apontam a viabilidade da utilização de mineração de dados para análise
preditiva de alunos em risco de evasão na UnB nos cursos da área de computação.
Como a metodologia utilizada não empregou nenhum conceito específico dessa área do
conhecimento, pode-se usá-la para outros cursos de graduação da UnB. 
