% definição do problema e abordagem atual
A evasão na Universidade de Brasília (UnB) é um problema grande, com consequências
acadêmicas, sociais e econômicas. A abordagem da UnB para resolução deste problema
consiste em separar os alunos em dois grupos: aqueles que estão em condição e
aqueles que não estão, e ter os alunos em condição orientados. 
% proposta de solução
Pensando em melhorar tal abordagem, a pesquisa em questão objetiva a concepção de um
sistema previsor capaz de indicar quais alunos estão com maior risco de não
conseguirem formar.  Desse modo, o sistema permitiria a UnB agir antes de um aluno
entrar em condição e agir de acordo com o risco de evasão apresentado por cada aluno.
% metodologia
Para o desenvolvimento do sistema previsor, dados descaracterizados de alunos de
graduação de cursos da área de computação que
ingressaram de 2000 até 2016 e já saíram da universidade foram utilizados.
Treinaram-se modelos de aprendizagem de máquina (\textit{Naive Bayes}, ANN, SVR, Regressor
Linear e Random Forests) e o desempenho deles foi analisado nos dados de teste. 
% resultados/conclusão
Os modelos de aprendizagem de máquina testados tiveram, em geral, bom desempenho. O
melhor desempenho foi o do regressor linear. Os resultados obtidos apontam a
viabilidade da utilização de aprendizagem de máquina para análise preditiva de alunos
em risco de evasão na UnB.

