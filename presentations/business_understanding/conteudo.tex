\section{Entendimento do Negócio}

\begin{frame}{Motivação e Definição do Problema}
    \begin{itemize}[itemsep=3ex]
        % custos da evasao 
        \item A evasão nas universidades já foi estudada por muitos autores, tratando-se de
        um grande problema que traz desperdícios acadêmicos, sociais e econômicos.
        % variedade do conceito de evasao
        \item O conceito de evasão varia, podendo-se considerar evasão do curso,
            evasão da instituição, evasão do sistema.
        \item No entender de especialistas, o conceito de evasão adotado depende da
            pesquisa \cite{mec_conceito_evasao}.
    \end{itemize}
\end{frame}

% causas da evasao 
\begin{frame}{Conceito de Evasão Adotado e Diversas Causas da Evasão}
    \begin{itemize}[itemsep=3ex]
        \item Para a pesquisa, considera-se evasão a saída do aluno do curso, não
            se faz distinção entre as causas da evasão
        \item Várias causas para a evasão são possíveis: desempenho acadêmico,
            mudança de cidade/região, mudança de curso, desligamento voluntário,
            problemas sociais, necessidade econômica de entrar no mercado de
            trabalho.
    \end{itemize}
\end{frame}

% Abordagem da UnB e seus problemas
\begin{frame}{Abordagem da UnB 
        \footnote{A UnB teve um prejuízo com evasão estimado em 95,6
        milhões, segundo o Correio Brasiliense}}
    \begin{itemize}[itemsep=3ex]
        \item De acordo com as normas internas da UnB, alunos em condição (definido
            mais adiante) devem procurar o orientador acadêmico.
        \item Assim, o esforço atual da UnB de evitar evasão consiste de separar os
            alunos em dois grupos (em condição ou não) e ter os alunos em condição
            orientados.
        \item Problema 1: Essa abordagem classifica alunos (uma amostra muito diversificada) em
            apenas dois grupos.
        \item Problema 2: Agir quando o aluno se encontra em condição pode ser tarde
            demais. Afinal, se o aluno não cumpre condição ele é desligado. 
    \end{itemize}
\end{frame}

% condicao na UnB
\begin{frame}{Critérios Para Condição na UnB}
    \begin{itemize}[itemsep=3ex]
        \item Ter duas reprovações na mesma disciplina obrigatória.
        \item Não ser aprovado em quatro disciplinas do curso em dois períodos
            regulares consecutivos.
        \item Chegar ao último período letivo no projeto pedagógico do curso sem a
            possibilidade de concluir.
    \end{itemize}
\end{frame}

% desligamento da unb
\begin{frame}{Critérios Para Desligamento na UnB}
    \begin{itemize}[itemsep=3ex]
        \item Durante dois semestres consecutivos não efetuar matrícula em disciplina. 
        \item Durante dois semestres consecutivos ser reprovado com SR em todas as
   disciplinas. 
        \item Não cumprir condição (detalhado mais adiante).
    \end{itemize}
\end{frame}

\begin{frame}{Cumprir Condição}
    \begin{itemize}[itemsep=3ex]
         \item Ser aprovado nas disciplinas obrigatórias anteriormente cursadas com
             duas reprovações. 
         \item Ser aprovado no mínimo de créditos do curso nos próximos dois períodos
             letivos subsequentes.
         \item Cumprir plano de estudo aprovado pela Comissão de Acompanhamento e
             Orientação (CAO).
    \end{itemize}
\end{frame}


% Minha proposta e benefícios
\begin{frame}{Apresentação da Proposta}
    \begin{itemize}[itemsep=3ex]
        \item Utilizar dados passados para criação de um sistema previsor de modo a 
            identificar alunos em risco de serem desligados. 
        \item Utilização de vários modelos de aprendizagem de máquina diferentes.
    \end{itemize}
\end{frame}

% adicionei dados passados
\begin{frame}{Dados disponíveis}
    \begin{itemize}[itemsep=3ex]
        \item Dados descaracterizados dos alunos de graduação da UnB, contendo tanto 
            informações de perfil dos alunos quanto forma de ingresso e menções nas
            matérias da UnB. 
    \end{itemize}
\end{frame}


\begin{frame}{Utilidade da Proposta}
    \begin{itemize}[itemsep=3ex]
        \item Permitir a universidade agir antes de um aluno entrar em condição. 
        \item Permitir a universidade agir com mais flexibilidade, 
            de maneira diferenciada de acordo com o risco apresentado por um aluno. 
            % alunos com maior risco recebem mais atencao, sem entretanto
            % negligenciar alunos com risco moderado
    \end{itemize}
\end{frame}

% Como avaliar a performance do sistema classificador 
\begin{frame}{Medindo o Desempenho}
    \begin{itemize}[itemsep=3ex]
        \item Separação dos dados em conjunto de treino e de teste, medição do
            desempenho do modelo em conjunto de teste.
        \item O modelo, para um estudante particular, ``acerta'' quando:
        \begin{itemize}[itemsep=1.5ex]
        \item Avalia o risco de um estudante ser desligado como sendo baixo 
            e o estudante não seja jubilado.
        \item Avalia o risco de um estudante ser desligado como sendo
            alto e o estudante termina sendo jubilado.
        \end{itemize}
    \end{itemize}
\end{frame}

\begin{frame}{Ponderando Tipos de Erro}
    \begin{itemize}[itemsep=3ex]
        \item Falso negativo - Sistema julga que o risco de um aluno ser jubilado é baixo, e
            o aluno acaba sendo jubilado
        \item Falso positivo - Sistema julga que o risco de um aluno ser jubilado é alto, e
            o aluno acaba não sendo jubilado
        \item Falsos negativos são mais graves que falsos positivos. 
            Logo, deve haver uma ponderação de modo a refletir isso.
        \item Escolher fator de ponderação me parece uma escolha a ser guiada pelo
            bom senso. Para a pesquisa, sugiro $$w = 3$$
    \end{itemize}
\end{frame}
