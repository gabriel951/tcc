% imputação
% limpeza dos dados
% histograma dos valores
% matriz de correlação

\section{Preparação dos Dados}
\begin{frame}{Seleção dos Dados}
\begin{itemize}[itemsep=3ex]
    \item Utilização dos dados de 2000 até 2009 como treino e de 2010 até 2015 para
        testes. 
    \item Atributos que serão levados em consideração foram explicados na fase de
        entendimento dos dados: sexo, idade, UF, cotista, tipo da escola, raça,
        curso, forma de ingresso, IRA.
\end{itemize}
\end{frame}

\begin{frame}{Construção dos Atributos Derivados}
\begin{itemize}[itemsep=3ex]
    \item Atributos derivados que serão construídos já foram definidos na fase
        anterior.
\end{itemize}
\end{frame}

\begin{frame}{Histograma da Distribuição dos Atributos Derivados - Coeficiente de
    Melhora Acadêmica}
\begin{itemize}[itemsep=3ex]
        \item <Por histograma aqui>
\end{itemize}
\end{frame}

\begin{frame}{Matriz de Correlação}
\begin{itemize}[itemsep=3ex]
    \item <Por matriz de correlação aqui>
    \item Qual valor de covariância é indicado de modo a eliminar feature
\end{itemize}
\end{frame}

\begin{frame}{Limpeza dos Dados}
\begin{itemize}[itemsep=3ex]
    \item Optou-se por não considerar features nos quais mais de 60\% das entradas
        fossem missing values.
    \item Descartou-se assim o feature raça.
\end{itemize}
\end{frame}

\begin{frame}{Limpeza dos Dados}
\begin{itemize}[itemsep=3ex]
    \item Para o caso de atributos que podem apresentar missing values, foi feita
        imputação (exemplo: coeficiente de melhora acadêmica).
\end{itemize}
\end{frame}

\begin{frame}{Integração dos Dados}
\begin{itemize}[itemsep=3ex]
    \item Dados originais da SIGRA foram integrados com a informação dos currículos
        antigos e novos dos cursos, de modo a saber quais matérias são de quais
        semestres.
\end{itemize}
\end{frame}

\begin{frame}{Transformação dos dados}
\begin{itemize}[itemsep=3ex]
    \item Dados textuais serão transformados em dados numéricos.
    \item IRA foi normalizado para 0 a 1, assim como o coeficiente de melhora
        acadêmica.
\end{itemize}
\end{frame}

