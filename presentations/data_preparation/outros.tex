\begin{frame}{Histograma - Razão entre disciplinas cursadas por semestre e
    disciplinas do curso}
\begin{itemize}[itemsep=3ex]
        \item <Por histograma aqui>
\end{itemize}
\end{frame}

\begin{frame}{Histograma - Razão entre aprovações em disciplinas obrigatórias cursadas 
        por semestre e disciplinas do curso}
\begin{itemize}[itemsep=3ex]
        \item <Por histograma aqui>
\end{itemize}
\end{frame}

\begin{frame}{Taxa de Aprovação nas Disciplinas mais difíceis do Semestre}
\begin{itemize}[itemsep=3ex]
        \item <Por histograma aqui>
\end{itemize}
\end{frame}

\begin{frame}{Demais Atributos Derivados}
\begin{itemize}[itemsep=3ex]
        \item Demais atributos derivados incluem: booleano que indica se um aluno
            está ou não em condição e 
            a posição em relação ao semestre (0 caso seja o pior aluno, 1 caso seja o
            melhor).
\end{itemize}
\end{frame}

\begin{frame}{PCA}
\begin{itemize}[itemsep=3ex]
    \item Utilização da técnica PCA (Principal Component Analysis) para verificar se
        features estão demasiadamente relacionados. 
\end{itemize}
\end{frame}

\begin{frame}{Limpeza dos Dados}
\begin{itemize}[itemsep=3ex]
    \item Optou-se por não considerar features nos quais mais de 60\% das entradas
        fossem missing values.
    \item Descartou-se assim o feature raça.
\end{itemize}
\end{frame}

\begin{frame}{Limpeza dos Dados}
\begin{itemize}[itemsep=3ex]
    \item Para o caso de atributos que podem apresentar missing values, foi feita
        imputação (exemplo: coeficiente de melhora acadêmica).
    \item Assim, caso haja missing value coloca-se a média do feature. 
\end{itemize}
\end{frame}

\begin{frame}{Integração dos Dados}
\begin{itemize}[itemsep=3ex]
    \item Dados originais da SIGRA foram integrados com a informação dos currículos
        antigos e novos dos cursos, de modo a saber quais matérias são obrigatórias. 
\end{itemize}
\end{frame}

\begin{frame}{Transformação dos dados}
\begin{itemize}[itemsep=3ex]
    \item Utilização de dummy variables para representar dados categóricos.  
\end{itemize}
\end{frame}

