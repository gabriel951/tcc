\section{Entendimento dos Dados}

% coletando dados iniciais
\begin{frame}{Coleta de Dados}
    \begin{itemize}[itemsep=3ex]
        \item Dados do SIGRA (Sistema de Graduação), obtidos em parceria com a UnB.
        \item Pesquisas prévias utilizaram o desempenho acadêmico do aluno
            \cite{adeodato}.
        \item Outros estudos apontam a necessidade de considerar também dados sociais
            \cite{predict_retention}.
        \item Como os dados contém tais informações, devem ser suficientes.
    \end{itemize}
\end{frame}

% divisao dos dados
\begin{frame}{Divisão dos Dados}
    \begin{itemize}[itemsep=3ex]
        \item Utilizar dados de 2000 até 2015. 
        \item Dados de treino: alunos que entraram depois de 2000 e saíram antes de
            2010.
        \item Dados de teste: alunos que entraram depois de 2010 e se formaram até
            2015.
    \end{itemize}
\end{frame}

% dados disponiveis a serem utilizados
\begin{frame}{Dados Brutos a Serem Utilizados}
    \begin{itemize}[itemsep=3ex]
        \item Dados sociais: sexo, idade, residente do local, cotista,
            tipo da escola, raça, curso, forma de ingresso.
        \item Desempenho acadêmico: IRA.
    \end{itemize}
\end{frame}

% dados derivados a serem utilizados
\begin{frame}{Dados Derivados a Serem Utilizados}
    \begin{itemize}[itemsep=3ex]
        \item coeficiente de melhora acadêmica (razão entre notas do primeiro semestre
            e notas do segundo semestre) 
        \item razão entre número de disciplinas reprovadas por semestre, trancadas
            por semestre e aprovadas por semestre frente ao total de disciplinas do
            semestre
        \item razão entre disciplinas cursadas por semestre e quantidade 
            de disciplinas do curso 
        \item razão entre disciplinas obrigatórias cursadas por semestre e quantidade 
            de disciplinas obrigatórias do curso (*)
        \item flag indicando aprovação ou não na disciplina mais difícil do semestre.
    \end{itemize}
\end{frame}

% propriedades gerais dos dados
\begin{frame}{Propriedades Gerais dos Dados}
    \begin{itemize}[itemsep=3ex]
        \item São 7 features relacionados às condições sociais dos estudantes e 11
            features relacionados ao desempenho acadêmico
        \item São x entradas na base de dados 
    \end{itemize}
\end{frame}


