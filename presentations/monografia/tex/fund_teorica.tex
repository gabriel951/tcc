% TODO: falar do modelo crips-dm, se quiser encher linguiça
Neste capítulo, descreve-se a fundamentação teórica necessária para compreender a
pesquisa. Assim, nas seções seguintes explica-se a problemática da evasão nas universidades
públicas, a abordagem da UnB para tal situação e como estatística descritiva e o
método de Kendall podem ser usados em mineração de dados. 

\section{Evasão nas Universidades} 
A evasão nas universidades nacionais é um problema que traz desperdícios acadêmicos,
sociais e econômicos, tendo sida estudada por diversos autores. Apesar disso, o
conceito de evasão pode variar \cite{mec_conceito_evasao} de acordo com a pesquisa,
já que pode-se considerar evasão do curso, evasão da instituição ou evasão do
sistema educacional. 

\section{Abordagem da UnB frente à Evasão}
\par Os critérios para que um aluno seja desligado na UnB são mostrados na Figura
\ref{desligamento}. 
\begin{figure}[!ht]
    \caption{Critérios Para Desligamento na UnB}
    \centering
    \includegraphics[width = 18cm]{desligamento.png}
    \label{desligamento}
\end{figure}
 
A abordagem da UnB para evitar desligamento consiste de separar os alunos em dois
grupos (alunos em condição e alunos que não estão em condição) e ter os alunos em
condição orientados. Os critérios para que um aluno esteja em condição são
\cite{manual_calouro}: 
\begin{itemize}
    \item Ter duas reprovações na mesma disciplina obrigatória
    \item Não ser aprovado em quatro disciplinas do curso em dois períodos regulares
        consecutivos
    \item Chegar ao último período letivo do curso sem a possibilidade de concluir
\end{itemize}

\section{Estatística Descritiva} 
Estatística descritiva é comumente usada em pesquisas de mineração de
dados, na fase de análise exploratória dos dados. A estatística descritiva permite
identificar problemas de qualidade nos dados, como por exemplo a identificação de
valores de atributos faltantes ou a identificação de \textit{outliers} ou a identificação de
atributos com cardinalidade irregular \cite{ml_book}. Técnicas comuns de estatística
descritiva incluem histogramas, gráficos de barra e \textit{boxplots}. 

\section{O Coeficiente de Correlação de Kendall}
%TODO: escrever mais um pouco aqui caso deseje
Quando se trabalha com modelos de aprendizagem de máquina com muitas variáveis, é
desejável saber o grau de correlação entre as variáveis. Caso a correlação entre
algum par de variáveis seja muito alta, a eliminação de uma delas pode acarretar em
um modelo mais simples de ser compreendido (sem prejudicar o desempenho).
\par Assim, pode-se usar um coeficiente de correlação para medir o grau de dependência
entre atributos e decidir pela eventual eliminação de algum. Existem vários testes
para se medir a correlação entre variáveis, como o coeficiente de Spearman e o
coeficiente de correlação de Kendall \cite{kendall}. 

