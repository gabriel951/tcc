% TODO: problema referenciar artigos unpublished?
Neste capítulo, faz-se uma introdução do trabalho por meio da definição do problema,
da descrição da estrutura geral do documento em questão e dos objetivos traçados.

\section{Definição do Problema}
% fala do problema
A evasão nas universidades brasileiras já foi estudada por diversos autores,
tratando-se de um problema que trás desperdícios acadêmicos, sociais e econômicos.
A UnB (Universidade de Brasília) não é exceção, sendo afetada significativamente pelo
problema \footnote{A UnB teve um prejuízo com evasão estimado em 96,5 milhões,
segundo o Correio Brasiliense \cite{correio}}.

% o q a unb faz
\par Atualmente a abordagem da UnB consiste de separar os seus alunos entre aqueles
que estão em condição e aqueles que não estão e ter os alunos em condição orientados
\cite{manual_calouro}. Essa abordagem, entretanto, apresenta problemas:
\begin{itemize}
    \item Os alunos (uma amostra muito diversificada) são classificados em apenas
dois grupos.
    \item A UnB age apenas quando o aluno já se encontra em condição, quando pode
já ser tarde demais.
\end{itemize}

\section{Proposta de Solução}
% minha solução
A pesquisa aqui descrita justifica-se como uma tentativa de melhorar a abordagem
atual da UnB para o problema da evasão. Propõe-se utilizar dados passados para
a criação de um sistema previsor que seja capaz de identificar alunos em risco de
serem desligados. O sistema previsor forneceria um valor entre 0 e 1, com valores
mais próximos de 0 indicando um baixo risco do aluno ser desligado e valores pertos de
1 indicando um risco maior. Para isso, vários modelos diferentes de aprendizagem de máquina
serão estudados. 
\par Os dados utilizados são dados descaracterizados de alunos de
graduação da UnB, contendo tanto informações de perfil quanto a forma de ingresso e
as menções nas matérias da UnB. Caso bem sucedido, o sistema previsor permitiria a
UnB agir antes de um aluno entrar em condição. Outra vantagem seria a possibilidade
de agir com mais flexibilidade, de acordo com o risco de evasão apresentado por cada
aluno.

\section{Objetivos}
São objetivos deste trabalho: 
\begin{enumerate}
    \item Descrever os resultados obtidos nas primeiras 3 fases do CRISP-DM.
    \item Apresentar cronograma para finalização do trabalho no primeiro semestre de
        2017.
\end{enumerate}

\section{Organização do Restante do Documento}
% organização do documento
\par Descreve-se a seguir a organização do restante do documento. 
Na seção 2 é descrita a metodologia
adotada para a pesquisa, e a metodologia a ser utilizada nas 
fases ainda não realizadas. Na seção 3 apresentam-se e discutem-se os resultados
obtidos até o momento enquanto que na seção 4 faz-se o planejamento acerca das etapas
restantes. Finalmente, na seção 5, apresenta-se a conclusão.
