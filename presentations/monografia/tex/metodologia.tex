Nesta capítulo, descreve-se a metodologia utilizada na pesquisa até aqui e a
metodologia a ser usada nas próximas etapas. Detalha-se como foi feito o levantamento
do estado da arte, a obtenção dos dados e utilização dos dados, a seleção de
atributos, a eliminação de outliers e a análise preliminar por meio de estatística
descritiva. 

\section{Levantamento do Estado da Arte}
Foi feito o levantamento do estado da arte através da leitura de diversos artigos, de
modo a entender quais fatores são importantes para a evasão \cite{adeodato}
\cite{hoed_fatores} \cite{dropout_finland}, como técnicas de
aprendizagem de máquina podem ser utilizadas para resolver o problema \cite{adeodato}
 \cite{data_mining_retention}.
e como pesquisas passadas trabalharam com os dados da UnB
\cite{manual_calouro} \cite{hoed_sobrevivencia}. 

\section{Obtenção e Utilização dos Dados}
Obtiveram-se informações descaracterizadas relativas aos dados sociais e ao
desempenho acadêmico de alunos da UnB. Todos os dados utilizados vieram de uma só
fonte, de modo que o comum problema encontrado na área de mineração de dados de
garantir a consistência dos dados entre as várias fontes não foi enfrentado. 
\par Optou-se por restringir a pesquisa apenas aos alunos que entraram a partir de
2000 e saíram até 2015. Para simplificar a análise, e trabalhar apenas com uma área
específica, apenas os seguintes cursos foram considerados: 
\begin{itemize}
    \item Ciência da Computação (Bacharelado)
    \item Ciência da Computação (Licenciatura)
    \item Engenharia da Computação 
    \item Engenharia de Software
    \item Engenharia de Redes
    \item Engenharia Mecatrônica
\end{itemize}

\section{Seleção de Atributos}
% TODO: falar que obrigatoriedade entre disciplinas poderia ficar para uma proxima
% fase do trabalho
Com base no levantamento do estado da arte feito, selecionou-se quais atributos
teriam melhor condição de serem significativos para que um aluno fosse ou não
desligado. Assim sendo, lista-se a seguir os atributos sociais considerados:
\begin{itemize}
        \item Sexo
        \item Idade
        \item Residente no local 
        \item Cotista 
        \item Tipo da Escola 
        \item Raça
        \item Forma de Ingresso
\end{itemize}

Além de dados sociais, utilizou-se os seguintes atributos (relativos ao desempenho
acadêmico): 
\begin{itemize}
    \item Curso
    \item IRA
    \item Coeficiente de Melhora Acadêmica (razão entre notas do último semestre e
        notas do penúltimo semestre)
    \item Taxa de Aprovação, Taxa de Reprovação e Taxa de Trancamento
    \item Razão entre créditos integralizados por semestre e quantidade de créditos
        do curso
    \item Taxa de aprovação na disciplina mais difícil de cada semestre
    \item Booleano para indicar se aluno está ou não em condição
    \item Posição em relação ao semestre
\end{itemize}
%TODO: se quiser encher linguiça, explicar o que são a taxa de aprovação, reprovação
%e trancamento
%TODO: se quiser encher linguiça, falar sobre o sistema de menções da unb
A posição em relação ao semestre $P$ para um determinado aluno é definida como sendo:
\begin{equation}
    P = \frac{N_m}{N_a}
\end{equation}
onde $N_m$ é o número de alunos com IRA menor que o do estudante em questão
(considerando apenas aqueles que entraram no mesmo curso do estudante, no mesmo ano e
no mesmo semestre) e $N_a$ é o número de alunos (novamente considerando apenas
aqueles que entraram no mesmo curso do estudante, no mesmo ano e no mesmo semestre). 

\section{Eliminação de Outliers}
Decidiu-se não trabalhar com casos de alunos que após ingressar na universidade não
demonstraram interesse em cursar matérias (por exemplo, aqueles que reprovaram em
todas as disciplinas com SR). Tais casos foram tratados como outliers.
Foi feita a eliminação de outliers do espaço amostral, após a análise individual de
cada caso. 

\section{Análise Preliminar por meio de Estatística Descritiva}
Foi feita uma análise preliminar por meio de estatística descritiva. Para cada 
atributo que apresentava uma distribuição contínua fez-se o seu histograma.
Já para atributos categóricos, fez-se o gráfico de barras. Depois disso, avaliou-se o
grau de relacionamento entre os diversos atributos que compunham o modelo por meio do
coeficiente de correlação de Kendall. 
