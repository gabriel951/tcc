Neste capítulo apresentam-se e analisam-se os resultados obtidos após se ter seguido a
metodologia explicada no capítulo anterior. Descreve-se os resultados após a
eliminação de Outliers, mostram-se os histogramas e gráficos de barra obtidos para
cada um dos atributos e os resultados do coeficiente de correlação de Kendall para a
relação entre variáveis. 

\section{Eliminação de Outliers}
Conforme dito anteriormente, a análise de quais alunos são outliers foi feita
estudando caso a caso os alunos que não conseguiram passar em nenhuma disciplina.
Eliminaram-se 203 estudantes do espaço amostral dessa maneira, ficando-se assim com
3751 amostras. 

\section{Histogramas e Gráficos de Barra Para Atributos}
Apresentam-se a seguir os histograms e gráficos de barra para os atributos que puderam
ser calculados. Como alguns atributos dependiam de informação acerca da quantidade de
créditos de cada matéria, e não tinha-se tal informação disponível no começo da
pesquisa, não foi possível gerar a estatística descritiva para tais atributos. Tais
atributos são descritos mais adiante, na seção \ref{atributos_problematicos}.

\subsection{Gráfico de Barra para Atributo Sexo}


\subsection{Atributos que não Puderam ser Calculados} \label{atributos_problematicos}

\section{Correlação de Kendall}
