% definição do problema e proposta de solução
A evasão na Universidade de Brasília (UnB) é um problema grande, com consequências
acadêmicas, sociais e econômicas. A abordagem da UnB para resolução deste problema
consiste em separar os alunos em dois grupos: aqueles que estão em condição e
aqueles que não estão, e ter os alunos em condição orientados. Pensando em melhorar
tal abordagem, a pesquisa em questão objetiva a concepção de um sistema previsor
capaz de indicar quais alunos estão com maior risco de serem desligados. 
O modelo de processos CRISP-DM é utilizado. 
% o objetivo desse trabalho eh. o que foi feito
O objetivo deste documento é registrar as tarefas feitas na parte inicial dessa
pesquisa e estabelecer um cronograma para a parte final. Na parte inicial,
traçou-se como meta cumprir as seguintes fases do CRISP-DM: entendimento do negócio,
entendimento dos dados e preparação dos dados. Já o cronograma para a parte final
inclui a etapa da modelagem e avaliação de desempenho dos modelos.
% conclusao
À exceção de não ter sido possível calcular alguns atributos, a parte inicial da
pesquisa foi feita conforme planejado. Por fim, o cronograma apresentado aponta a
viabilidade da execução de toda parte final da pesquisa no próximo semestre.  
