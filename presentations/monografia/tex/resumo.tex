% definição do problema e proposta de solução
A evasão na Universidade de Brasília é um problema grande, com consequências
acadêmicas, sociais e econômicas. A abordagem da UnB na resolução deste problema
consiste em separar os alunos em dois grupos: aqueles que estão em condição e
aqueles que não estão, e ter os alunos em condição orientados. Pensando em melhorar
tal abordagem, a pesquisa em questão objetiva a concepção de um sistema previsor
capaz de indicar quais alunos estão com maior risco de serem desligados. Nessa
pesquisa, optou-se por utilizar o modelo de processos CRISP-DM. 

% o objetivo desse trabalho eh. o que foi feito
O objetivo deste documento é documentar a parte inicial dessa pesquisa e estabelecer
um cronograma para a parte final. Na parte inicial, foi feito o entendimento do
negócio, o entendimento dos dados e a preparação dos dados. O cronograma para a
parte final inclui a etapa da modelagem e avaliação de desempenho dos modelos.
% conclusao
Apesar de não ter sido possível calcular alguns atributos, a parte inicial da
pesquisa foi feita conforme planejado e o cronograma apresentado aponta a
viabilidade da execução de toda parte final da pesquisa no próximo semestre.  
